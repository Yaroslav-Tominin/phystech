

\documentclass[a4paper,12pt]{article} 



\usepackage[T2A]{fontenc}			
\usepackage[utf8]{inputenc}			 
\usepackage[english,russian]{babel}	


% Математика
\usepackage{amsmath,amsfonts,amssymb,amsthm,mathtools} 


\usepackage{wasysym}


\author{Томинин Ярослав, 778 группа}
\title{Домашнее задание №2}
\date{\today}


\begin{document} 

\maketitle
\newpage


1.\\ 
Поймем, что 
\[\frac{n^{\frac{5}{2}}}{4 \sqrt2} \leqslant 
\frac{n^{\frac{5}{2}}}{4 \sqrt2} \sqrt{1+\frac{8}{n^2}+\frac{40}{n^3}} \leqslant
\frac{n}{2} \sqrt{\frac{n^3}{8}+n+5} \leqslant
\sum_{1\dots{n}} \sqrt{i^3+2i+5} \leqslant
n^{\frac{5}{2}}\sqrt{1+\frac{2}{n^2}+\frac{5}{n^3}} \leqslant
8 n^{\frac{5}{2}}
\]
Отсюда видно, что $h(n)=\Theta(n^{\frac{5}{2}})$\\
2.\\
Поймем, что \[\exists N : 2^n \leqslant (3+o(1))^n\leqslant 4^n \]
Отсюда следует, что \[n \leqslant 
log_2 {2^n+\Theta(n^{100})}\leqslant
log_2 f(n) \leqslant
log_2 {4^n+\Theta(n^{100})}\leqslant
2n+log_2 {\frac{\Theta(n^{100})}{2^n}}\leqslant
2n+1
\]
В одном неравенстве был  использован тот факт, что показательная функция при достаточно больших n растет быстрее, чем степенная.\\
Видим, что в общем случае верно $log{f(x)= \Theta (n)}$\\
3.\\
Поймем, что общее количество слов можно задать такой формулой
\[\sum_{b=1}^{\sqrt{n}} \sum_{i=0}^{ b} (log_2 n + \sum_{j=0}^{\frac{i}{2}} 1) \leqslant
\sum_{b=1}^{\sqrt{n}} \sum_{i=0}^{ b} (log_2 n + \frac{(i+1)}{2})\leqslant
\sum_{b=1}^{\sqrt{n}} (b \log_2 n + b\frac{(b+1)}{2})\leqslant
n \log_2 n + n^{\frac{3}{2}}= O( n^{\frac{3}{2}})
\]\\
С друой стороны
\[\sum_{b=1}^{\sqrt{n}} \sum_{i=0}^{ b} (log_2 n + \sum_{j=0}^{\frac{i}{2}} 1) \geq
\sum_{b=1}^{\sqrt{n}} \sum_{i=0}^{ b} (log_2 n + \frac{i}{2})\geq
\sum_{b=1}^{\sqrt{n}} (\frac{b}{2} \log_2 n + \frac {b}{2}\frac{(\frac  {b}{2}+1)}{2})\geq
\frac{n}{4} \log_2 n + \frac{n^{\frac{3}{2}}}{8}= \Omega( n^{\frac{3}{2}})
\]\\
Следовательно $f(x)= \Theta (n^{\frac{3}{2}})$\\
4.\\
а)\\
238x+385y=133;\\
$238=2*7*17$\\
$385=5*7*11$\\
$133=19*7$\\
$x_1=\frac{x}{19}$\\
$y_1=\frac{y}{19}$\\
Тогда получим $238x_1+385y_1=7;$\\
Если сократим $34x_1+55y_1=1;$
a=34, b=55\\
$a_a=1,a_b=0,b_a=0,b_b=1$\\
a=34, b=21\\
$a_a=1,a_b=0,b_a=-1,b_b=1$\\
a=13, b=21\\
$a_a=2,a_b=-1,b_a=-1,b_b=1$\\
a=13,b=8\\
$a_a=2,a_b=-1,b_a=-3,b_b=2$\\
a=5,b=8\\
$a_a=5,a_b=-3,b_a=-3,b_b=2$\\
a=5,b=3\\
$a_a=5,a_b=-3,b_a=-8,b_b=5$\\
a=2,b=3\\
$a_a=13,a_b=-8,b_a=-8,b_b=5$\\
a=2,b=1\\
$a_a=13,a_b=-8,b_a=-21,b_b=13$\\
a=0,b=1\\
$a_a=55,a_b=-34,b_a=-21,b_b=13$\\
При $x_1=34$, а $y_1=-21$ результат правильный.\\ Следовательно $x=646$, а $y_1=-399$\\
б)\\
143x+121y=52\\
143=11*13\\
121=11*11\\
52=4*13\\
Так как нод двух делящихся на 11 чисел делится на 11, то 
52 должно делиться на 11, а оно не делиться на 11. \\
5.\\
%функция Divide(x, y)\\
%Вход n битовые x и y , причём y   1.\\
 %Выход: частное и остаток от деления x на y.\\
 %если x=0: вернуть (q,r)=(0,0)\\
%(q, r) ← Divide(x/2, y)\\
%q←2·q, r←2·r\\
%если x нечётно: r←r+1\\
%еслиr y: r←r−y,q←q+1\\
%вернуть (q, r)\\ 



Проведем аналогию нашего алгоритма с делением двоичных чисел по столбику.
Наш алгорим отбрасывает последнюю цифру от x до тех пор, пока он не станет равен 0. После этого мы принимаем r и q равными за 0 и начинаем "развертывать" наш алгорим. Мы перехолим на n-1 шаг рекурсии с нулевыми q и r, а x у нас равен первой цифре числа. Теперь поймем, что каждый раз после проведения наших операций q и r будут равны значениям деления первых n-i цифр числа x на y. Докажем это по индукции.\\
 База у нас уже есть, действительно, ведь при i=n q и r равны 0.\\
 Переход. Если в q и r содержаться остатки при делении первых n-i цифр. То по алгоритму мы умножаем q и r на 2 и прибавляем 1, если x нечетно. Нетрудно заметить, что у нас сохраняется равенство $x_{n-i+1}=q_{n-i}*2*y+r_{n-i}*2+(x_{n-i+1} mod 2)$ Так как $x_{n-i+1}=x_{n-i}*2+(x_{n-i+1} mod 2)$ и $x_{n-i}=q_{n-i}*y+r_{n-i}$. Последнее следует из предположения индукции. После замены $q=q_{n-i}*2$, $r_{n-i}*2+(x_{n-i+1} mod 2)$ могло оказаться так, что r>=y.(Поймем, что изначально r был строго меньше y. Следовательно 2r-1<2y ). Поэтому, если r>y, то мы r=r-y, q=q+1. После этого r<y(только что доказали).Тогда индукция доказана. Следовательно алгоритм работает корректно.\\
 Получим оценку на время работы. Наш алгорим выполняет nрекурсивных вызовов(так как в двоичной записи число x занимает n бит). В каждом выозове у нас могут выполняться операции сложения, сравнения и сдвига влево. Это занимает O(n). Поэтому всего времени $O(n^2)$\\
 6.\\
 Приведем нашу дробь к правильному виду. Далее за a обозначим ее числитель, а за b знаменатель. После этого проделаем следующие действия \[h_i=\lfloor \frac{a_i}{b}\rfloor\]\[ a_{i+1}=a_i*2-\lfloor \frac{a_i}{b}\rfloor\]
 Грубо говоря мы просто будем выполнять деление в столбик, а в нашем массиве h на i месте будет находиться число, стоящее на i-ом разряде. Так же в другом массиве мы будем хранить все $a_i$Теперь поймем,что наш алгоритм корректен и мы всегда сможем найти период. Сначала докажем, что всегда найдется такой момент, что $a_{i+k}=a_i$. Докажем это от противного. Допустим это не так, тогда поймем что число a может изменяться от 0 до b-1. Следовательно через b шагов у нас точно найдутся одинаковые a. (Обозначим их за$a_i, a_{j}$)Следовательно мы нйдем период, который будет совпадать с $h_i ... h_{j-1}$\\
 Асимтотика. \\
 Поймем, что в каждый раз наши операции деления и вычитания стоили $O(n^2)$, а сами наши операции повторялись максимум b раз. Мы знаем, что $b<=2^{n+1}$. Следовательно асимтотика нашего алгоритма $O(n^2* 2^{n})$
7.\\
 
\end{document}