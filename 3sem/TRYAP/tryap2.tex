
\documentclass[a4paper,12pt]{article} 




\usepackage[utf8]{inputenc}			 
\usepackage[english,russian]{babel}	
\usepackage[all]{xy}

%автомат
\usepackage{tikz}
\usetikzlibrary{arrows,automata}


% Математика
\usepackage{amsmath,amsfonts,amssymb,amsthm,mathtools} 


\usepackage{wasysym}


\usepackage{graphicx}
\graphicspath{{pictures_tryap1/}}



\author{Томинин Ярослав, 778 группа}
\title{Домашнее задание №8}
\date{\today}


\begin{document} 

\maketitle
\newpage
\textbf{1.}\\
1) Поймем, что $L \subseteq T$. Так как любое слово из L составленно прибавлением к раз каких-то из 2) к одному из 1). Так как любое слово из 1) содержит меньше 3 b, а при прибавлении любого слова из 2) справа стоит a), то у нас тоже не образуется 3 подряд идущих b.\\
Докажем, что $T \subseteq L$: рассмотрим произвольное слово из T и допустим, что оно не содержится в L(доказываем от противного). Но ведь мы можем составить это слово по нашим правилам: просто будем прибавлять а пока не останется всего одной а до буквы b(если изначально была одна, то ничего не прибавляем, если если идет сразу b то ставим b или две b). После этого в первом случае мы прибавляем  аbb или ab в зависимости сколько будет подряд идущих b. Посде этого в обоих случаях будет идти а и мы будем действовать так, как в первм случае. Корректность: изначально есть две возможности, либо сначала идет несколько b(тогда мы добавляем эти b и сводим задачу к 1 случаю), или идет а(тогда мы добавляем а, пока не останется одной а). Когда мы дошли до этого !!!мы можем утверждать что после очередной а пойлет или одна b или две b, после которых обязательно пойдет а. Поэтому наш алгоритм работает корректно.\\
2)\\
$  \lbrace bba,ba,a\rbrace ^* \cdot \lbrace \epsilon | b|bb\rbrace$\\
Изначально мы выбираем либо e, либо b, либо bb. А потом к началу можем добавить  один из трех вариантов. Ого да это же как раз правило постоения L и мы уже доказали, что это правило порождает именно наш язык.\\
3)
\begin{center} 
\begin{tikzpicture}[>=stealth',shorten >=1pt,auto,node distance=2cm] 
\node[initial, state,accepting] (q_1) {$q_1$}; 
\node[state,accepting] (q_2) [right of=q_1] {$q_2$}; 
\node[state,accepting] (q_3) [right of=q_2] {$q_3$}; 

\path[->] (q_1) edge [loop above] node {a} (q_1) 
edge [bend left] node {b} (q_2) 
(q_2) edge node {b} (q_3) 
edge [bend left] node {a} (q_1) 
(q_3) edge [bend left] node {a} (q_1) 
; 

\end{tikzpicture} 
\end{center}
Докажем корректность нашего автомата: докажем, что все выводимые слова имеют не более 2 подряд идущих b. Дайствительно, ведь находясь в состоянии  1 мы уверены, что до этого точно идет буква a, а находясь в состоянии 2 мы уверены, что до этого стоит ровно одна  b, потому что мы пришли из состояния 1 по b.  Аналогично в состоянии 3: мы пришли из 2 по b, следовательно у нас 2 b. То есть в каждом состоянии есть инвариант по количеству b, так как все состояния допустимые и в каждом состоянии количество подряд идущих b не больше 3, то мы доказади наше утверждение. Докажем, что все слова, не содержащие более чем 3 b подряд мы можем вывести. Рассмотрим слово и предположим, что в нем нет трех подряд идущих b, но оно не выводимо. Тогда мы будем находится в состоянии 1 и наш автомат будет поддерживать наш инвариант, потому что слово может не быть выведенно если мы находимся в недопустимом состоянии(но все состояния допустимы) или наш автомат сломался(но он может сломаться только тогда, когда он находится в 3 состоянии и мы подаем на вход b, но по нашему инварианту у нас 2 подряд идущих b и b  мы подавать уже не можем). Следовательно мы можем составить любое слово.\\
2.\\
a)\\
\begin{center} 
\begin{tikzpicture}[>=stealth',shorten >=1pt,auto,node distance=2cm] 
\node[initial, state,accepting] (q_1) {$q_1$}; 
\node[state] (q_2) [right of=q_1] {$q_2$}; 

\path[->] (q_1) edge [loop above] node {1} (q_1) 
edge [bend left] node {0} (q_2) 
(q_2) edge [loop above] node {1} (q_2) 
edge [bend left] node {0} (q_1) 

; 

\end{tikzpicture} 
\end{center}

Докажем корректность автомата: докажем, что существует инвариант: в состоянии 1 четное количество 0, а в состоянии 2 - нечетное. Докажем это по индукции\\
\textbf{База:} сначала мы находимся в состоянии 1 и у нас ноль нолей. \\
\textbf{Переход:} По нашему предположению в 1 всегда четное количество нолей, если мы попадаем в 1, то мы переходим из 2 состояния, но во втором состоянии мы могли оказаться только из 1. По предположению в 1 было четное колиество 0, мы переходим во 2(будет нечетное) и переходим в 1 (будет опять четное).\\
Следовательно в допустимом состоянии будет всегда четное количество 0. А так как во 2 мы можем попасть только из 1, то во втором всегда нечетное количество 0.\\
Докажем, что вссе слова с четным количеством 0 принадлежат нашему автомату. Мы знаем наш инвариант, следовательно автомат не может находиться во 2 состоянии, поэтому, если мы будем решать от противного, то мы сможем только предположить что автомат сломался. Но как он мог сломаться, если из каждого состояния у него есть стрелка на любой случай жизни? Противоречие.\\
b)\\
\begin{center} 
\begin{tikzpicture}[>=stealth',shorten >=1pt,auto,node distance=2cm] 
\node[initial, state] (q_1) {$q_1$}; 
\node[state,accepting] (q_2) [right of=q_1] {$q_2$}; 

\path[->] (q_1) edge [loop above] node {0} (q_1) 
edge [bend left] node {1} (q_2) 
(q_2) edge [loop above] node {0} (q_2) 
edge [bend left] node {1} (q_1) 

; 

\end{tikzpicture} 
\end{center}
Воспользуемся рассуждениями из предыдущего пункта. Мы доказали, что в 1 состоянии будет всегда четное число 0(в данном случае четное число 1), а в состоянии 2 нечетное число 0(в данном случае нечетное число 1). Поэтому в допустимом состоянии все наши слова будут иметь нечетное число 1. Аналогично предыдущему все слова с нечетным количеством едениц будут принадлежать нашему автомату. От противного. Если слово не принадлежит, то либо оно находится в недопустимом состоянии(но мы доказали инвариант, поэтому такого быть не может), либо автомат сломался(но такого тоже не может быть, потому что из каждого состояния у него есть стрелка на любой случай жизни). Противоречие.\\

в) Постоим в задаче 3.\\
\textbf{3.}\\
\begin{center} 
\begin{tikzpicture}[>=stealth',shorten >=1pt,auto,node distance=2cm] 
\node[initial, state] (A) {$q_{11}$}; 
\node[state,accepting] (B) [below of=A] {$q_{12}$};
\node[state] (C)[right of =A] {$q_{21}$}; 
\node[state] (D) [below of=C] {$q_{22}$}; 


\path[->] (A) edge [bend left] node {1} (B) 
edge [bend left] node {0} (C) 
(B) edge [bend left] node {1} (A) 
(B) edge [bend left] node {0} (D) 
(C) edge [bend left] node {1} (D) 
(C) edge [bend left] node {0} (A) 
(D) edge [bend left] node {1} (C)
(D) edge [bend left] node {0} (B)  

;
\end{tikzpicture} 
\end{center}
Здесь мы использовали конструкцию произведения. За состояния отвечает $q_{ij}$, где i-состояние в 1 автомате, а j-во втором. Видно, что из состояния 11 автомат переходит в 12, если мы подаем 1 (так как 1 автомат остается в том же состоянии, а второй автомат переходит в 2) . Аналогично делаются все преходы. Когда мы находимся в состоянии ij и нам подают k, то  мы смотрим на состояние первого автомата q, после перехода к, на состояние второго автомата d, и делаем вывод, что состояние нового автомата qd. Допустимым положением, будет положение, в котором 2 автомата будут находиться в допустимом положении-12.\\
\textbf{4.}\\
Обозначим состояния автомата за $q_{ab}$, где а-состояние 1 автомата, а b-состояние второго автомата. Если мы посмотрим на 1 автомат и состояние a будет допускающим, то это слово допустимо для 1. Следовательно, если мы слелаем эту модификацию, то когда мы окажемся в допускающем состоянии нашего нового автомата, то либо1, либо 2 автомат будеи в допускающем состоянии. Следовательно все слова содержатся в объединении языков. Докадем, что все слова из объежинения мы можем получить и завершим доказательство. От противного: найдем слово, которое принадлежит одному из автоматов. После того, как мы его подадим на вход, один из автоматов бужет в допустимом состоянии. Поэтому для нового это будет тоже допустимое состояние. Противоречие.\\



\end{document}
