
\documentclass[a4paper,12pt]{article} 




\usepackage[utf8]{inputenc}			 
\usepackage[english,russian]{babel}	
\usepackage[all]{xy}

%автомат
\usepackage{tikz}
\usetikzlibrary{arrows,automata}


% Математика
\usepackage{amsmath,amsfonts,amssymb,amsthm,mathtools} 


\usepackage{wasysym}


\usepackage{graphicx}
\graphicspath{{pictures_tryap1/}}



\author{Томинин Ярослав, 778 группа}
\title{Домашнее задание №10}
\date{\today}


\begin{document} 

\maketitle
\newpage
\textbf{1.}\\
Рассмотрим наш язык. Разделим его на два множества:\\
\textbf{1) Слова, в которых есть нарушение(слово содержит одно из подслов ba,cb,ca)}\\
\textbf{2) Слова, которые имеют вид $a^kb^zc^d$, причем одновременно k,z,d не равны друг другу}\\
\textbf{Утверждение:} Каждое слово из языка принадлежит хотя бы одному множеству(факт того, что слово не принадлежит одновременно двум множествам оставим без доказательста, считая, что он сразу следует из построения)\\
\textbf{Доказательство:} От противного. Предположим, что нашлось такое слово из языка, которое не принадлежит ни одному множеству, тогда в нем нет нарушений и мы его можем представить в виде $a^kb^zc^d$, но это означает, что оно принадлежит второму множеству. Противоречие.\\
Рассмотрим следующие подмножества: подмножество состоит из слов вида$a^kb^zc^d$, где k>z(k>d,z>k,z>d,d>k,d>z). Эти шесть множест могут пересекаться друг с другом, но главное, что из объединение дает наше 2 множество.\\
Построим грамматики для каждого из шести подмножеств.\\
\textbf{Грамматика$ G_1,k>z$}
\begin{align*}
	S_1&\to A_1B_1C_1\\\
	A_1&\to aA_1|a\\\
	B_1&\to aB_1b|\epsilon\\\
	C_1&\to cC_1|\epsilon\\\
\end{align*}
\textbf{Утвержение:} данная грамматика выводит множество, состоящее из слов вида$a^kb^zc^d$, где k>z.\\
\textbf{Доказательство:} эта грамматика является конкатенацией трех грамматик, которые выводят языки $a^k, k>0; a^qb^q,q \geq 0; c^t,t \geq 0$. Поэтому их конкатенация выводит нужное множество.\\
Данное утверждение можно расширить на z>d,k<z,z<d грамматики, поэтому дли них мы не будем это доказывать.\\
\textbf{Грамматика$ G_2,k>d$}
\begin{align*}
	S_2&\to A_2C_2\\\
	A_2&\to aA_2|a\\\
	B_2&\to B_2b|\epsilon\\\
	C_2&\to aC_2c|\epsilon|B_2\\\
\end{align*}
Идея этой грамматики заключается в том,что мы порождаем слово $a^ka^mCc^m, k>0$ и потом можем заменить нетерминал C на B и вывести произвольное количество букв B. Эта грамматика выводит все слова из нашего 2 подмножества и при этом только их, так как нет нарушения и букв а точно больше, чем букв с. Это рассуждение распространяется на оставшееся множество.
\textbf{Грамматика$ G_3,z>k$}
\begin{align*}
	S_3&\to A_3B_3C_3\\\
	A_3&\to aA_3b|\epsilon\\\
	B_3&\to B_3b|b\\\
	C_3&\to cC_3|\epsilon\\\
\end{align*}
\textbf{Грамматика$ G_4,z>d$}
\begin{align*}
	S_4&\to A_4B_4C_4\\\
	A_4&\to aA_4|\epsilon\\\
	B_4&\to B_4b|b\\\
	C_4&\to bC_4c|\epsilon\\\
\end{align*}
\textbf{Грамматика$ G_5,d>k$}
\begin{align*}
	S_5&\to A_5C_5\\\
	A_5&\to aA_5c|\epsilon|B_5\\\
	B_5&\to B_5b|\epsilon\\\
	C_5&\to C_5c|c\\\
\end{align*}
\textbf{Грамматика$ G_6,d>z$}
\begin{align*}
	S_6&\to A_6B_6C_6\\\
	A_6&\to aA_6|\epsilon\\\
	B_6&\to bB_6c|\epsilon\\\
	C_6&\to C_6c|c\\\
\end{align*}
Построим грамматику для первого множества:\\
\textbf{Грамматика$ G_7$, существует нарушение}
\begin{align*}
	S_7&\to A_7baA_7|A_7caA_7|A_7cbA_7\\\
	A_7&\to aA_7|bA_7|cA_7|\epsilon\\\
\end{align*}
Эта грамматика является конкатенацией произвольного слова нарушения и произвольного слова, поэтому это и есть наше множество 1.\\
В результате получим грамматику:\\
\textbf{Грамматика$ G$}
\begin{align*}
	S&\to S_1|S_2|S_3|S_4|S_5|S_6|S_7\\\
	S_1&\to A_1B_1C_1\\\
	A_1&\to aA_1|a\\\
	B_1&\to aB_1b|\epsilon\\\
	C_1&\to cC_1|\epsilon\\\
	S_2&\to A_2C_2\\\
	A_2&\to aA_2|a\\\
	B_2&\to B_2b|\epsilon\\\
	C_2&\to aC_2c|\epsilon|B_2\\\
	S_3&\to A_3B_3C_3\\\
	A_3&\to aA_3b|\epsilon\\\
	B_3&\to B_3b|b\\\
	C_3&\to cC_3|\epsilon\\\
	S_4&\to A_4B_4C_4\\\
	A_4&\to aA_4|\epsilon\\\
	B_4&\to B_4b|b\\\
	C_4&\to bC_4c|\epsilon\\\
	S_5&\to A_5C_5\\\
	A_5&\to aA_5c|\epsilon|B_5\\\
	B_5&\to B_5b|\epsilon\\\
	C_5&\to C_5c|c\\\
	S_6&\to A_6B_6C_6\\\
	A_6&\to aA_6|\epsilon\\\
	B_6&\to bB_6c|\epsilon\\\
	C_6&\to C_6c|c\\\
	S_7&\to A_7baA_7|A_7caA_7|A_7cbA_7\\\
	A_7&\to aA_7|bA_7|cA_7|\epsilon\\\
\end{align*}
\textbf{2.}\\
Да, это КС-язык. Для того, чтобы это понять, достаточно его переписать в виде:\\
$\lbrace ^nb^nb^mc^m|n,m \geq 0\rbrace$\\
Мы знаем, что для языка $a^nb^n$ грамматика выглядит так:\\
\textbf{Грамматика$ G_1$}
\begin{align*}
	S_1&\to aS_1b|\epsilon\\\
\end{align*}
Для языка $b^mc^m$ грамматика выглядит так:\\
\textbf{Грамматика$ G_2$}
\begin{align*}
	S_2&\to bS_2c|\epsilon\\\
\end{align*}
Тогда их конкатенация и есть наш язык:\\
\textbf{Грамматика$ G$}
\begin{align*}
	S&\to S_1S_2\\\
	S_1&\to aS_1b|\epsilon\\\
	S_2&\to bS_2c|\epsilon\\\
\end{align*}
\textbf{3.}\\
1)Поймем, что $A\setminus R=A\cap\overline{R}$. Так как регулярные языки замкнуты относительно дополнения и при этом мы можем строить кострукцию произведения для регулярного языка и КС-языка. Исходя из этого, полученный язык будет КС-языком.\\
2)Нет, это не верно. \\
\textbf{Контрпример:} $R=\sum^*$, $A=\overline{B},B={uu|u\in \sum^* }$. Так как мы доказывали, что А-КС-язык и мы так же доказывали, что B не является КС-языком. Результат будет равен B, который  не является КС-языком.\\
3)Да, верно. Так как это КС-язык, то рассмотрим грамматику, если все слова $\alpha$, находящиеся в грамматике справа заменить на $\alpha^R$, то мы получим грамматику, которая будет выводить $A^R$.\\
\textbf{4.}\\
Напишем отрицание леммы о накачке:\\
\[\forall p \exists w \in L: |w|\geq p, \forall x,u,y,v,z : |uyv|\leq p; |uv|>0, \exists i\geq 0: w_i=xu^iyv^iz \notin L\]
Возьмем для произвольного p слово $w=a^{2p}a^pb^pa^{2p}$\\
Разбиением будем называть uyv.\\
Разбиение этого слова может лежать в $a^{2p}$, тогда мы накачаем несколько а и так как количество букв изменится, то слово x будет содержать только а, а $x^R$(на самом деле это не $x^R$, мы просто называем так третью часть слова) справа точно будет содержать b, поэтому слово не принадлежит языку.\\
Если разбиение лежит между $a^{2p}$ и $a^{p}$, то проделаем то же самое и получим слово не из языка.\\
Если разбиение между $a^{p}$ и $b^{p}$, то возьмем i=2, тогда длина увеличится меньше , чем на p, поэтому х будет содержать только а, а $x^R$(на самом деле это не $x^R$, мы просто называем так третью часть слова) справа точно будет содержать b, поэтому слово не принадлежит языку.\\
Если разбиение между $b^{p}$ и $a^{p}$, то тогда разбиение имеет вид:$a^{2p}a^pb^quyva^{w}$.\\
Возможны следующие случаи:\\
\textbf{u содержит и b, и a(при этом v пустое)}\\
\textbf{v содержит и b, и a(при этом u пустое)}\\
\textbf{в u и v содержатся либо только а, либо только b}\\
\textbf{Обоснуем:} если не выполняется третье и в u,v не могут быть одновременно а и b(так как есть только один переход от b к а),  то получается, что либо u, либо v -пустое слово, а другое содержит и a, и b.(так как оба пустых быть не может по лемме о накачке) А это и есть 1,2 утверждения. То есть все возможные варианты лежат в этих трех случаях.\\
В первом случае возьмем i=3p(длина каждой трети станет не меньше, чем 4p), тогда заметим, что количество букв а справа до первой буквы b равно p(причем эта буква b точно входит в третью треть слова). А количество букв а слева до первой b равно 3p. Следовательно, мы нашли слово, которое не принадлежит языку.\\
Во втором случае проделаем все то же самое и прийдем к тому же противоречию.\\
В третьем случае, если u,v содержат одинаковые буквы, то в случае, если они оба содержат а, выберем i=6p(длина каждой трети станет не меньше, чем 4p), тогда в правой трети будут только а, а в левой после 3p cимволов а будет стоять b. Поэтому слово будет не принадлежать языку. Если u,v содержат буквы b, то выберем i=6p(длина каждой трети станет не меньше, чем 4p), тогда в правой части будет идти p букв а, а потом бужет стоять b, а левой части первыми 3p буквыми будут а. Поэтому слово не принадлежит языку.\\
Рассмотрим случай, когда в u,v по отдельности содержат только один тип символов, но при этом они содержат разные символы. Тогда мы точно знаем, что u содержит  b, а v содержит а. Наше слово имеет вид $a^{2p}a^pb^{p-q}b^qa^{2p-r}a^r,|u|=q;|v|=r$\\
Тогда , выбирая i=kp+1, получим слово $a^{2p}a^pb^{p(1+kq)}a^{p(2+kr)}$
Если r=0, то q точно больше 0, и если мы возьмем k=6, то длина слова будет не меньше, чем 12p, заметим, что в правой трети справа стоит 2p букв а, а потом b, а в левой трети слева стоит 3p букв а, а потом b. Поэтому слово не принадлежит языку.\\
Если $r\neq0$, то выберем k=6p,то длина слова будет не меньше, чем 12p, и правая треть состоит только из а, а левая содержит b. Следовательно слово не принадлежит языку.\\
Следовательно язык не является регулярным.\\

\textbf{5.}\\
Вычислим FIRST\\
\begin{tabular}{ || l | l | l | l | l | l || }
\hline
$F_i$ &E & T & F & E' & T'  \\ \hline
$F_0$ &$\oslash$ & $\oslash$ & $\oslash$ & $\epsilon$ & $\epsilon$  \\ \hline
$F_1$ &$\oslash$ & $\oslash$ & (,id & $\epsilon,+$ & $\times, \epsilon$  \\ \hline
$F_2$ &$\oslash$ & (,id & (,id & $\epsilon,+$ &   $\times, \epsilon$ \\ \hline
$F_3$ &$(,id$ & (,id & (,id & $\epsilon,+$ &   $\times, \epsilon$ \\ \hline
\hline
\end{tabular}
\\
Теперь вычислим FOLLOW\\
\begin{tabular}{ || l | l | l | l | l | l || }
\hline
$F_i$ &E & T & F & E' & T'  \\ \hline
$F_0$ &$\oslash$ & $\oslash$ & $\oslash$ & $\oslash$ & $\oslash$  \\ \hline
$F_1$ &$)$ & $+$ &$\times$ & $\oslash$ & $\oslash$  \\ \hline
$F_2$ &$)$ & +,) & $\times,+$ & $)$ &   $+$ \\ \hline
$F_3$ &$)$ & +,) & $\times,+,)$ & $)$ &   $+,)$ \\ \hline
$F_4$ &$)$ & +,) & $\times,+,)$ & $)$ &   $+,)$ \\ \hline
\hline
\end{tabular}
\end{document}