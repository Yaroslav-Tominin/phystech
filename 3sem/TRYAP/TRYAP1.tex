
\documentclass[a4paper,12pt]{article} 




\usepackage[utf8]{inputenc}			 
\usepackage[english,russian]{babel}	
\usepackage[all]{xy}

%автомат
\usepackage{tikz}
\usetikzlibrary{arrows,automata}


% Математика
\usepackage{amsmath,amsfonts,amssymb,amsthm,mathtools} 


\usepackage{wasysym}


\usepackage{graphicx}
\graphicspath{{pictures_tryap1/}}



\author{Томинин Ярослав, 778 группа}
\title{Домашнее задание №8}
\date{\today}


\begin{document} 

\maketitle
\newpage
\textbf{1.}\\
1) Заметим, что после декартового произведения мы получим пары, в которых на первом месте будут стоять элементы из первого множества, а на втором из второго. Так как первое множество не содержит элемента b, то данное нам утверждение не верно.\\
2) Ответ: |A|x|B|. Обратим внимание, что для любого элемента из А существует |B| соседей, а элементов из А |А|.\\
3)По определению декартого произведения все элементы полученного множества содержат на втором месте элемент из пустого множества. Но так как в пустом множестве нет элементов, то и в полученном множестве тоже нет элементов.\\
\textbf{2.}\\
1)1+5+4+3+2+1=16\\
Рассмотрим пустые подслова-их 1 штука\\
Рассмотрим подслова длины 1- их 5\\
Рассмотрим подслова длины 2- их 4\\
Рассмотрим подслова длины 3- их 3\\
Рассмотрим подслова длины 4- их 2\\
Рассмотрим подслова длины 5- их 1\\
Итого:16 подслов\\
2)\\
a)5\\
b)3\\
c)2\\
d)По определению это количество вхождений подслова $\epsilon$ в наше слово. Другими словами нам нужно найти количество подслов, имеющих различное i(где i - номер буквы, после которой он стоит). Я утверждаю, что их 6. i=0,1,2,3,4,5. Для начала заметим, что это различные подслова, так как они находятся в разных местах. Докажем, почему больше нет подслов от противного: допустим, что есть, тогда есть два рядом стоящих $\epsilon$, тогда их индексы совпадают и они стоят на одном и том же месте. Поражение.\\
Следовательно, всего 6 подслов.\\
3)\\
Нет, $\epsilon$ нельзя представить так в слове $aa\epsilon b$\\
\textbf{3.}\\
При конкатенации слова нечетной длины со словом нечетной длины получается слово четной длины. Поэтому Все слова будут четной длины, докажем, что это будут все четнын числа: всевозможные конкатенации первого элемента из 1 множества и всех элементов из второго множества дают нужный результат(если не считать ноль четным числом).\\
Поэтому результат такой: {$\lbrace a^{2n} | n>0,n\in N\rbrace$}\\
\textbf{4.}\\
a)\\
$(a\cup b)^*\cdot(a \cdot b \cup b\cdot a) \cdot (a\cup b)^*$\\
Поймем, что если слово содержит а и б, то эти две буквы должны встретиться рядом. (иначе все слово будет состоять из одной и той же буквы). Тогда подслово аб или ба должно входить в любое наше слово. То есть любое наше слово состоит из произвольного перфикса(который мы получим так $(a\cup b)^*$ ) , нашего подслова аб или ба и произвольного суфикса $(a\cup b)^*$ \\
b)\\
Имеем U.ab.V
Поймем, что если U будет содержать a, после которой будет идти b, то наше слово не будет подходить под условие. Так же если V будет содержать после a букву b, то наше слово тоже не будет подхолить под условие. Поэтому имеем: $b^*\cdot a^* \cdot a \cdot b \cdot b^* \cdot a^*$\\
В этом языке есть все подходящие слова, потому что сначала идет сколько-то б(возможно 0), а после того, как встречается а, по нашему условию могут идти только а. Потом мы встечаем аб и идет сколько-то б(возможно 0), а после того, как встречается а, по нашему условию могут идти только а. Это и написано в РВ.\\
Осталось доказать почему все эти слова подходят: сначала у нас идет слово из букв b  какой-то длины(возможно 0), потом идут а, потом идет ab(это как раз первая пара), потом b, а после них а(здесь нет пар).\\
в)\\
Если после а будет идти b, то наше слово не будет походить. Поэтому имеем:$b^*\cdot a^*$\\
В этом языке есть все подходящие слова, потому что сначала идет сколько-то б(возможно 0), а после того, как встречается а, по нашему условию могут идти только а. В точности это и написано в РВ.\\
Докажем, что эти слова подходят под условие: понятно, что после а не будет б, поэтому не может быть аб.\\
\textbf{5.}\\\\
\includegraphics[width=1.0\textwidth]{z}
Докажем его корректность: когда мы находимся в начале и принимаем а, то понимаем, что он потенциально может быть частью ааб, и двигаемся дальше(если это б, то тут все понятно, остаемся на месте). Далее при получении а мы сдвигаемся дальше, если же это б, то слигаемся в самый конец и начинаем все с начала. Далее, если мы получаем а, то мы остаемся на месте(тк мы собрали уже 2 а и дожидаемся b), если же это б, то мы попадаем в конец и после рандомного добавления а и б можем все закончить. То есть нам для того, чтобы дойти до выхода нам точно нужно собрать ааб, с другой стороны, если в слове есть ааб, то мы окажемся у выхода и в любой момент можем закончить.\\
\begin{center} 
\begin{tikzpicture}[>=stealth',shorten >=1pt,auto,node distance=2cm] 
\node[initial, state] (q_1) {$q_1$}; 
\node[state] (q_2) [right of=q_1] {$q_2$}; 
\node[state] (q_3) [right of=q_2] {$q_3$}; 
\node[state, accepting] (q_4) [right of=q_3] {$q_4$}; 

\path[->] (q_1) edge [loop above] node {b} (q_1) 
edge [bend left] node {a} (q_2) 
(q_2) edge node {a} (q_3) 
edge [bend left] node {b} (q_1) 
(q_3) edge [loop above] node {a} (q_3) 
edge node {b} (q_4) 
(q_4) edge [loop above] node {a, b} (q_4); 

\end{tikzpicture} 
\end{center}
\end{document}
