
\documentclass[a4paper,12pt]{article} 




\usepackage[utf8]{inputenc}			 
\usepackage[english,russian]{babel}	
\usepackage[all]{xy}

%автомат
\usepackage{tikz}
\usetikzlibrary{arrows,automata}


% Математика
\usepackage{amsmath,amsfonts,amssymb,amsthm,mathtools} 


\usepackage{wasysym}


\usepackage{graphicx}
\graphicspath{{pictures_tryap1/}}



\author{Томинин Ярослав, 778 группа}
\title{Домашнее задание №7}
\date{\today}


\begin{document} 

\maketitle
\newpage
\textbf{1.}\\
Создадим грамматику по следующему алгоритму: \\ 
\textbf{Нетерминальный символ} будет соответствовать состоянию в нашем языке S-0,A-1,B-2,C-3,D-4.\\
\textbf{Аксиомой} будет S(соответствует 0 состоянию). \\
\textbf{Правила вывода:} Слева у нас будут Нетерминальные символы, правило вывода будет соответствовать переходу по терминальному символу, слева будет стоять символ, соответствующий состоянию, из которого мы переходим, а справа буква, по которой мы перешли и символ, соответствующий состоянию, в которе мы перешли. Так же еще будет правило, связанное с допускающими состояниями: Если состояние допустимо, то $A \rightarrow \epsilon$(A-терминал, соответствующий допустимому состоянию).\\
Для нашего автомата грамматика будет следующей:\\
$S\rightarrow A|aC$\\
$A\rightarrow bC|aB$\\
$B\rightarrow aC|\epsilon$\\
$C\rightarrow bD$\\
$D\rightarrow aS|S|\epsilon$\\
\textbf{Корректность:} Докажем, что $L(G) \subset L$. Заметим, что в нашем слове может быть не более одного нетерминала(так как мы один нетерминал переводим в $\alpha$ либо содержащу 1 нетерминал(правило вывода 1 типа), либо не содержащую нетерминалов(правило вывода 2 типа)). Поэтому если мы пользуемся правилом вывода 2 типа, то это слово принадлежит L(G)(потому что в нем был 1 нетерминальный символ и мы его убрали). Докажем, что нетерминальный символ в$ \alpha$ cоответствует состоянию автомата.(в которое он попал при прочтении слова, стоящего слева от терминального символа) По индукции: \\
\textbf{База:} в начале S cоответствует начальному состоянию.\\
\textbf{Переход:} $\alpha$ содержит нетерминал, соответствующий состоянию автомата, посмотрим на правила вывода для этого терминала, по построению наша грамматика содержит правила вывода, соответствующие переходу из этого состояния и правило вывода 2 типа(если это состояние допускающее). Заметим, что правила вывода 1 типа поддерживают инвариант, так как $\alpha=w X$, а после применения вывода 1 $\beta = wz Y$, где Y соответствует состоянию, в которое перейдет автомат после прочтения символа z.(по построению).\\
Заметим, что мы доказали, что если слово w принадлежит L, то мы сможем вывести $\alpha=wX$, где X-нетерминал, соответствующий принимающему состоянию, поэтому$w \in  L(G)$. Так же для любого слова $w \in  L(G)$ предыдущее правило вывода будет содержать нетерминал, соответствующий принимающему состоянию автомата. Поэтому $w\in L$. Поэтому эти яхыки совпадают.\\




\textbf{2.}\\
В предыдущей задаче мы ставили нетерминальные символы в соотвествие состояниям. В этой задаче попробуем сделать наоборот. S-0,A-1,B-2. 
\textbf{Алгоритм:} подправим нашу грамматику так, чтобы она имела вид $A\rightarrow wN$, для этого заметим, что правило вывода $A\rightarrow aa$(1) эквивалентно $A\rightarrow aaN$ и $N \rightarrow \epsilon$(2)\\
\textbf{Доказательство:} заметим, что в нашей грамматике соблюдается инвариант:в нашем слове может быть не более одного нетерминала(доказывается так же, как и в предыдущей задаче). Наши слова имеют вид $\alpha=wA$,поэтому, когда мы используем правило (1), то мы получаем слово $x=waa$, если же мы заменим (1) на (2), то мы получим то же самое. Докажем, почему не возникнет других слов. Заметим, что нетерминал N мы можем получить только из А, причем нетерминал N мы можем заменить только на $\epsilon$. Поэтому допустимым словом можем быть только слово вида $xaa$, где $\alpha=wA$, но такие слова мы могли получить и с помощью (1) правила вывода. \\
Теперь обобщим доказательство 1 задачи на правила вида\\
$A\rightarrow wB$(1), где w уже является подсловом.\\
\textbf{Лемма:} в нашей грамматике правило вида $A\rightarrow wB$, где $w=w_1...w_n$ можно заменить на эквивалентные правила $A\rightarrow w_1A_1$,...,$A_{n-1}\rightarrow w_nB$(2)\\
\textbf{Доказательство:} Заметим, что все слова, получающиеся из (1) могут получитбся из второго тривиальным образом. Так же заметим, что если мы используем нетерминалы $A_1,...,A_{n-1}$(то есть одно из правил вывода (2)), то это означает, что у нас будет слово вида $\beta=ywB$, где $\alpha=yA$: заметим, что $A_i$ нетерминал мы могли получить только из $A_{i-1}$ (при i>1)(так как для них существует всего лишь одно правило вывода). Следовательно, правила вывода мы сможем начать применять, если у нас бужет $\alpha=yA$, поэтому все слова,  в которых используется правило из (2) имеют вид $\beta=ywB$, но эти слова мы можем получить с помощью (1). Поэтому эти правила вывода эквивалентны.\\
Тогда теперь мы свели нашу задачу к предыдущей. Построим ДКА, соглассно полученному алгоритму:\\


\begin{center} 
\begin{tikzpicture}[>=stealth',shorten >=1pt,auto,node distance=2cm] 

\node[initial,state,accepting] (S)  {$S$}; 
\node[state] (S_aA)[above of=S]  {$S_{aA}$}; 
\node[state] (S_bA) [right of=S_aA]  {$S_{bA}$}; 




\node[state] (A) [ right of=S] {$A$}; 


\node[state] (S_aB)[below of=A]  {$S_{aB}$}; 

\node[state] (B) [ right of=A] {$B$}; 
\node[state,accepting] (X) [ above right  of=B] {$X$}; 


\path[->] (S)  edge node {$a$} (S_aA);
\path[->] (S_aA)  edge node {$b$} (S_bA);
\path[->] (S_bA)  edge node {$a$} (A);


\path[->] (S)  [bend right]edge node {$a$} (S_aB);
\path[->] (S_aB)[bend right]  edge node {$b$} (B);



\path[->] (A)  [bend left]edge node {$a$} (B);
\path[->] (A)  [bend left]edge node {$aa$} (X);

\path[->] (B)  edge node {$b$} (A);
\path[->] (B)  [bend left]edge node {$a$} (S);








\end{tikzpicture} 
\end{center}

\textbf{3.}\\
Нет, не является. Рассмотрим слово abaaa. Для него существутет два вывода(заметим что любой вывод в нашей грамматике является одновременно и левым и правым).\\
1)\\
$\underline{S}$\\
$S\rightarrow aS_{aA}$\\
$a\underline{S_{aA}}$\\
$S_{aA} \rightarrow bS_{bA}$\\
$ab\underline{S_{bA}}$\\
$S_{bA} \rightarrow aA$\\
$aba\underline{A}$\\
$A \rightarrow aaX$\\
$abaaa\underline{X}$\\
$X \rightarrow \epsilon$\\
$abaaa$\\
2)
$\underline{S}$\\
$S\rightarrow aS_{aA}$\\
$a\underline{S_{aA}}$\\
$S_{aA} \rightarrow bS_{bA}$\\
$ab\underline{S_{bA}}$\\
$S_{bA} \rightarrow aA$\\
$aba\underline{A}$\\
$A \rightarrow aX$\\
$abaa\underline{B}$\\
$B \rightarrow a$\\
$abaaa\underline{S}$\\
$S \rightarrow \epsilon$\\
$abaaa$\\
\textbf{4.}\\
Докажем по индукции, что L(G) - язык всех возможных слов нечетной длины, у которых в середине стоит b.\\
Сделаем это по индукции по длине слова:\\
\textbf{База:} для n=1 все хорошо, так как существует два слова а,b ж; но мы предполагаем, что b не выводимо.\\
\textbf{Переход:} допустим, что для 2n-1 предположение выполнено, тогда докажем, что и для 2n+1 выполнено. Рассмотрим произвольное слово длины 2n+1, подходящее под условие(не содержит b в середине). Тогда уберем у него буквы на n и n+2 местах($w_{n},w_{n+2}$). Это слово длины 2n-1 содержится в L(G). Рассмотрим вывод этого слова и поймем, что последним правилом может быть только $S\rightarrow a$. Заметим еще одну фичу: это правило может быть использованно только в конце(так как в каждом правиле количество нетерминальных символов не увеличивается, а изначально оно ьыло равно 1, если мы используем это правило, то количество нетерминалов будет равно 0), поэтому нетерминал S на последнем правиле вывода находился посередине(изначально он находился в центре, а все правила, которые не понижают количество нетерминалов оставляют его в центре). Тогда заменим нетерминал S на $w_{n}Sw_{n+2}$(такое правило у нас точно есть, так как у нас всевозможные комбинации а,b). После этого заменим S на a и получим желаемое слово.\\
Этим мы локазали, что L-язык всех возможных слов нечетной длины, содержится в L(G). Обратное включение доказывается легче. Мы уже знаем, что терминал S всегда стоит в середине, поймем(пользуясь тем фактом, что мы можем воспользоваться правилом вывода, понижающим нетерминальные символы только в конце), что количество терминальных символов в слове четно, так как мы всегда добывляем по 2(не включая последний шаг выывода). В конце мы прибавляем 1 букву, причем S меняем на а, поэтому слово нечетной длины и в середине стоит а.\\
Осталось ответить на вопрос регулярный ли язык L.\\
Допустим, что он регулярный, тогда существует ДКА с p состояниями. Рассмотрим слово $b^{p+1}ab^{p+1}$ Тогда при прочтении $b^i,b^j$ наш автомат окажется в одном и том де состоянии, поэтому выкинем j-i букв b. Полученное слово допускается нашим автоматом, но слово не принадлежит языку. Противоречие, поэтому язык не регулярный.\\
2)\\
Предположим, что его дополнение регулярно, тогда дополнение дополнения тоже регулярно, но это и есть наш нерегулярный язык. Противоречие.\\
Поэтому дополнение тоже нерегулярно.\\
\textbf{5.}\\
a)\\
$S\rightarrow aSa|bSb|a|b|\epsilon$
Докажем, что $L(G)\subset PAL$\\
Аналогично предыдущим задачам можем доказать(так как в нашей грамматике нет правил, которые увеличивают количество нетерминалов), что правила 3,4,5 могут использоваться только в конце вывода. Рассмотрим вывод произвольного слова кроме последнего вывода, заметим, что S находится в середине(доказывается аналогично предыдущей задаче), причем, если выкинуть S, то слово будет палиндромом(в начале вывода оно пустое-палиндром, в процессе вывода мы можем использовать только 1,2 правило, которые сохраняют палиндромность). Отлично, заметим, что в конце вывода мы можем поменять слово S,которое стоит в середине только a,b,$\epsilon$, которые тоже не изменят его палиндромности.\\
Докажем, что $PAL \subset L(G)$ \\
Рассмотрим произвольное слово и напишем его вывод, если крайние буквы b, то первый вывод $S\rightarrow bSb$, иначе $S\rightarrow aSa$. Теперь откидываем эти буквы и смотрим на следующие. Делаем так, пока не останется либо ни одной буквы(тогда $S\rightarrow \epsilon$), либо одна буква $w_n$(тогда $S\rightarrow w_n$).\\
Доказали, что наша грамматика является ответом.\\
b)\\
$S\rightarrow aSb|\epsilon|aSbb$\\
Докажем, что $L(G)\subset L$\\
\textbf{Мини-лемма}: для любого слова из L(G) вывод будет заканчиваться 2 правилом(это уже много раз использовали и доказывали в предыдущих задачах). Причем, S будет разделять буквы а от b(база: S, пререход: можем использовать либо 1,либо 3 вывод,которые не нарушат предположение). Так же букв b будет не меньше, чем а, и их будет больше не более, чем в два раза(так как до последнего вывода мы можем польховаться лишь  1,3 правилами, которые увеличивают а на 1, а b либо на 2, либо на 1). Именно поэтому к концу вывода у нас будет такое слово, у которого S будет разделять а от b и, если мы выкинем S, то это слово будет принадлежать L. Но последний вывод как раз и убирает нетерминал. Поэтому мы доказали первое включение.\\
Докажем, что $L \subset L(G)$\\
Рассмотрим произвольное слово из L, в нем к букв а и 2k-i букв b, причем i не больше к. Применим 3 правило к-i раз, а потом применим 1 правило i раз. Тогда получим к букв а и 2k-i букв b.\\
Доказали, что наша грамматика сооответствует языку L.\\
в)\\
\begin{align*}
	S&\to aN|Kb|QbaQ\\\
	N&\to aNb|aN|\epsilon\\\
	K&\to aKb|Kb|\epsilon\\\
	Q&\to aQ|bQ|\epsilon\\\
\end{align*}
Докажем, что $L(G)\subset L$\\
Для этого достаточно показать, что в L(G) нет слова вида $a^nb^n$. Допустим обратное, тогда рассмотрим его вывод. Поймем, что baQ|Qba|bQa мы не могли использовать, так как тогда бы был переход от b к а, а у нас слово другого вида. Следовательно, мы воспользуемся либо 1, либо 2 правилом. Если используем 1, то потом мы можем применять правила из 2 строки, так как из нетерминала N не получается других нетерминалов. Заметим, что изначально количество букв а будет больше, чем b и эта разность будет точно не убывать. Аналогично со 2 правилом, тогда мы сможем использовать правила из 3 строки и разность букв b и a всегда будет больше 0.\\
Заметим, что мы хотели получить слова, разность букв в котором равна 0. Но полным перебором мы не смогли его получить. Следовательно, наш факт доказан.\\
Докажем, что $L \subset L(G)$\\
Рассмотрим произвольное слово из L, если в нем есть переход от b к a, то используем 3 правило. Тогда мы сможем прибавлять в начало произвольные символы(на i шаге смотрим на $w_i$ и меняем первую Q на $w_iQ$ ). В конце меняем первую Q на пустое слово. То же самое со второй Q: будем добавлять поочереди буквы из суффикса. В конце получим желаемый результат. \\
Если в слове нет перехода от b к а. Если в нем больше а, то сделаем 1 вывод, далее воспользуемся k раз(где к- количество букв b ) правилом aNb , потом добъем буквы а, которых не хватает.\\
Аналогично, если букв b больше. Используем 2 правило, далее используем правило aKb q раз(где q-количество букв а). Потом добавим недостающие b. \\
Мы доказали, что можем составить вывод для любого слова из L\\
Cледовательно, грамматика соответствует заявленному языку.\\





\end{document}