
\documentclass[a4paper,12pt]{article} 




\usepackage[utf8]{inputenc}			 
\usepackage[english,russian]{babel}	
\usepackage[all]{xy}

%автомат
\usepackage{tikz}
\usetikzlibrary{arrows,automata}


% Математика
\usepackage{amsmath,amsfonts,amssymb,amsthm,mathtools} 


\usepackage{wasysym}


\usepackage{graphicx}
\graphicspath{{pictures_tryap1/}}



\author{Томинин Ярослав, 778 группа}
\title{Домашнее задание №6}
\date{\today}


\begin{document} 

\maketitle
\newpage
\textbf{1.}\\
a)Предположим, что это не так. Тогда 
\[\exists x,y \in L_q : x  !\sim_L y\] 
Поэтому $\exists z: xz \in L , yz \notin L$\\
Но тогда состояние, в которое мы попадем, прочитав xz,yz, будет одинаковым. И оно должно одновременно являться как допускающим, так и не допускающим. Поэтому мы пришли к противоречию. Следовательно, $x \in L_q \Rightarrow L_q \subseteq [x]$ \\
б)Так как количество состояний в ДКА конечное число, то мы можем рассмотеть языки $L_q$. \\
Выберем из них те, в которых $\exists y \in L_q : y \in [x]$\\
Тогда мы уже знаем, что $y \in L_q \Rightarrow L_q \subseteq [y] \equiv [x]$\\
То есть мы сначала выбрали объединение множеств, которые точно содержат [x], а потом доказали, что они не содержат слов, которые не принадлежат x.\\
в) Рассмотрим $L_{qi}:\bigcup_{i} L_{qi}=[x]$\\
Допустим, что выполняется 1, но не выполняется 2, тогда $L_p,L_q \subseteq [x]; R_q \neq R_p$\\
$\exists z: az \in L, bz \notin L; b \in L_p$ где либо $a \in L_q, b \in L_p$, либо $a \in L_p, b \in L_q$(это следствие отрицания второго условия)\\
Тогда $a,b \in [x]$, так как выполняется 1. И мы пришли к противоречию, так как $\exists z: az \in L, bz \notin L; b \in L_p$ означает, что мы пришли к противоречию.\\
Допустим, что выполняется 2, но не выполняется 1.  Мы знаем, что $\exists y \in L_p, z : yz \in L, xz \notin L$ или $\exists  y \in L_p, z : xz \in L, yz \notin L$(из отрицания 1 условия)\\
Отлично, но заметим, что если мы начинаем из состояний p,q, то у нас не могло получится так, что пройди по одной и той же последовательности букв z мы бы оказались для одного в принимающем, а для лругого нет.(это следует напрямую из 2 условия). Поэтому мы пришли к противоречию.\\
\textbf{2.}\\
Рассмотрим $L \cap \overline{R}$ Мы получим язык, который отличается от $L_1$ на конечное количество слов. По конструкции пересечения мы умеем строить ДКА, так же мы умеем брать строить ДКА по отрицанию регулярного языка. Тогда полученный язык регулярный. Рассмотрим слова, на которые отличается наш язык от $L_1$.  Пострим ЛКА Ахо-Корасик по этому словарю и применим конструкцию объелинения. Мы получим регулярный язык. Именно поэтому он не мог быть не регулярным.\\
\textbf{3.}\\
Распределим числа по множествам: числа, дающие остаток 0 при делении на 3, числа, дающие остаток 1 при делении на 3, числа, дающие остаток 2 при делении на 3. Поймем, что:\\
1) Элементы в каждом множестве попарно эквивалентны: действительно, ведь достаточно проследить как меняется остаток при дописывании подслова, так как изначально он был одинаков, то и после операций умножения на 2 и прибавлений 1 он будет одинаковый. Следовательно они попарно эквивалентны.\\
2) Покажем, что для произвольные элементы из двух множеств не эквиваленты: достаточно просто предложить такой z для каждой из пар множеств. \\
0 и 1,z=0\\
0 и 2,z=0\\
1 и 2,z=1\\
Теперь мы знаем, что они разлечимы\\
3)Доказательство того, что мы каждое слово принадлежит одному подмножеству тривиально.\\
Так как классов эквивалентности 3, то язык регулярный.\\
\textbf{4.}\\
a)\\
Рассмотрим лва различных слова и докажем что они принадлежат различным классам эквивалентности.\\
$\exists z: xz \in L,yz \notin L$\\
Не нарушая общности предположим, что $|x| \leq |y|$\\
Рассмотрим z=$x^R$, тогда $xz \in L$, а yz может принадлежать и не принедлежать. Если он не принадлежит, то мы победили, иначе подберем другое z следующим образом: в этом случае $|x| < |y|$ так как в противном случае x=y, а мы выбрали разные слова. Тогда $y=xww^R$. Выберем новый z: если вначале w стоит а, то z=bx, иначе z=ax. Тогда $yz \notin L$\\
b)\\
Докажем, что существует 3 класса эквивалентности: слова, состоящие из какого-то количества букв b(может и нулевого), и потом из какого-то количества букв а(ненулевого); слова, состоящие из какого-то количества букв b(может и нулевого);
остальные слова(эти слова уже содержат аb. Докажем это от противного, допустим, что ксть слово, не содержащее ab, тогда либо оно пустое, либо состоит из а, либо состоит из b, дибо состоит из какого-то количества b, а потом а. Но эти все слова мы уже рахобрали, поэтому там нет таких слов и мы пришли к противоречию). \\
Мы доказали, что все слова принадлежат только одному множеству. Докажем, что слова, принадлежащие одному множеству попарно эквивалентны: для 1 множества предположим, что мы нашли различающий z для каких-то двух слов из этого множества. Тогда это означает, что xz содержит ab(следовательно и z содержит ab),  но yz не содержит ab(следовательно и z не содержит). Приходим к противоречию.\\
Для второго множества предположим, что мы нашли отличающий z, тогда xz содержит ab (следовательно z содержит b), а yz не содержит ab(следовательно z не содержит b). Противоречие.\\
Для третьего множества поймем, что какое z  мы ни будем приписывать, то все равно оба слова будут принадлежать нашему языку.\\
Докажем, что для каждой пары множеств существует различающий z.\\
Для 1 и 2 множества z=b\\
2 и 3 z=$\epsilon$\\
1 и 3 z=$\epsilon$\\
\begin{center} 
\begin{tikzpicture}[>=stealth',shorten >=1pt,auto,node distance=2cm] 

\node[initial,state] (2)  {$2$}; 
\node[state] (1) [ right of=2]{$1$}; 
\node[state,accepting] (3) [ right of=1] {$3$}; 

\path[->] (2)  [bend left]edge node {$a$} (1);
\path[->] (2)  [loop above]edge node {$b$} (2);
\path[->] (1)  [loop above]edge node {$a$} (1);
\path[->] (1)  [bend left]edge node {$b$} (3);
\path[->] (3)  [loop above]edge node {$a,b$} (3);





\end{tikzpicture} 
\end{center}
\textbf{5.}\\
a)\\
Поймем, что это язык всех слов, которые содержат хотя бы одну букву  а, так как мы можем выбрать x=w, y=$\epsilon$.\\
А для такого языка мы уже строили автомат.\\
\begin{center} 
\begin{tikzpicture}[>=stealth',shorten >=1pt,auto,node distance=2cm] 

\node[initial,state] (1)  {$1$}; 
\node[state,accepting] (2) [ right of=1] {$2$}; 

\path[->] (1)  [bend left]edge node {$a$} (2);
\path[->] (1)  [loop above]edge node {$b$} (1);
\path[->] (2)  [loop above]edge node {$a,b$} (2);







\end{tikzpicture} 
\end{center}
Поэтоиу этот язык регулярный.\\
b)\\
Разделим слова на следующие подмножества: слова, которые не содержат букву b и |w|=k, слова, которые содержат букву b : x=wbv и |w|-|v|=i;(w не содержит b, это ее первое вхождение)\\
Докажем, что все элементы каждого множества попарно эквивалентны.\\
Рассмотрим первый тип множеств. Допустим, что нашлось различающее слово для двух слов из подмножества, тогда $xz \in L, yz \notin L$. Но так как |x|=|y|, то должно существовать разбиение слова yz, которое удовлетворяет условиям языка. И мы пришли к противоречию.\\
Рассмотрим второй тип множеств. Допустим, что нашлось различающее слово для двух слов из подмножества, тогда $xz \in L, yz \notin L$. Но мы знаем, что $xz=w_1bv_1z$,$yz=w_2bv_2z$, где $|w_1|-|v_1|=|w_2|-|v_2|$. Поэтому слово yz должно принадлежать языку, так как $|w_1|+1<|z|+|v_1|$,следовательно,$|w_2|+1<|z|+|v_2|$. И мы пришли к противоречию.\\
Докажем, что существует различающее слово для каждой пары множеств\\
Для первого типа множеств с i<j это слово $b\underbrace{a...a}_{i+2}$, тогда слово из множества j при объединении с различающим будет не принадлежать нашему языку, а при объединении с i принадлежать.\\
Для множеств второго типа i,j различающим словом будет z=$\underbrace{a...a}_{i+2}$, тогда слово из множества j не будет принадлежать нашему языку, а слово из i будет.\\
Осталось показать, что все слова содержаьбся в одном множестве. Это достаточно очевидно, потому что если это не так, то длина слова будет принимать одновременно два разных значения.\\
\textbf{6.}\\
\textbf{Лемма 1}: автомат языка L(ДКА) является минимальным тогда и только тогда, когда множества R(q) различны.\\
Достаточность условия различности: допустим, что ДКА не минимален, тогда существует два эквивалентных состояния и это означает, что у них одинаковые R(q). Противоречие.\\
Необходимость условия различности: если мы совместим два состояния с одинаковыми R(q), то наш язык не изменится(можно доказать от противного, допустим, что нашлось новое слово, тогда есть правое слово, которого раньше не входило в R(q), но так как R(q) совпадали, то такого не могло произойти), поэтому оно необходимо.\\


\textbf{Лемма 2}: множество левых языков r(L) является множеством правых языков L, причем R(q) для L=L(q) для r(L)\\
Доказательство: допустим, что существует слово из R(q), которое не принадлежит L(q) языка r(L), тогда рассмотрим слово из L, содержащее это правое слово, его обращение содержится в r(L), поэтому это слово входит в L(r(L)). Допустим, что существует слово из L(q) языка r(L), которое не принадлежит R(q), тогда существует слово w, обращение которго должно принадлежать L, но тогда начальное слово должно принадлежать R(q). Противоречие.\\


\textbf{Лемма 2.0}: нужно в точности повторить те же самые рассуждения\\

\textbf{Лемма 3}: множество правых языков d(L) есть объединение правых языков состояний НКА, которые содержатся в этом состоянии ДКА.\\
Доказательство: рассмотрим состояние ДКА, по построению это состояние соответствует множеству состояний НКА, поэтому правый язык состояния ДКА это объединение правых языков множества состояний НКА.\\

Теперь рассмотрим язык drdr(L). Поймем, что он детерменированный, так как последняя операция d. Далее поймем, что этот ДКА принимает язык L: операция d не меняет язык, а двойная операция r возвращает нам тот же язык. Теперь рассмотрим множество левых языкок dr(L). Мы знаем, что они не пересекаются(иначе бы это было одно и то же состояние), тогда по лемме 2.0 множество правых языков явтомата rdr(L) не пересекается. Соответственно для drdr(L) множнство правых языков не пересекается, так как язык каждого состояния по 3 лемме является объединением правых языков состояний НКА(которые не пересекаются друг с другом), которые содержатся в этом состоянии ДКА. Тогда этот ДКА по 3 лемме является минимальным.\\


\end{document}