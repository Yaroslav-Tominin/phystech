
\documentclass[a4paper,12pt]{article} 




\usepackage[utf8]{inputenc}			 
\usepackage[english,russian]{babel}	
\usepackage[all]{xy}

%автомат
\usepackage{tikz}
\usetikzlibrary{arrows,automata}


% Математика
\usepackage{amsmath,amsfonts,amssymb,amsthm,mathtools} 


\usepackage{wasysym}


\usepackage{graphicx}
\graphicspath{{pictures_tryap1/}}



\author{Томинин Ярослав, 778 группа}
\title{Домашнее задание №4}
\date{\today}


\begin{document} 

\maketitle
\newpage
\textbf{1.}\\
\begin{center} 
\begin{tikzpicture}[>=stealth',shorten >=1pt,auto,node distance=2cm] 
\node[initial, state] (q_0) {$q_0$}; 
\node[state] (q_1) [below right of=q_0]{$b$}; 
\node[ state] (q_2) [below right of=q_1]{$ba$}; 
\node[ state] (q_3) [below right of=q_2]{$bab$}; 
\node[state] (q_4) [below right of=q_3] {$babb$}; 
\node[state] (q_5) [below right of=q_4] {$babba$}; 
\node[state] (q_6) [below right of=q_5] {$babbab$}; 
\node[state] (q_7) [below right of=q_6] {$babbaba$}; 
\node[state,accepting] (q_8) [below right of=q_7] {$babbabab$}; 

\path[->] (q_0) 
edge  node {$b$} (q_1) 
(q_0) 
edge [loop above] node {$a$} (q_0) 
(q_1) 
edge  [loop above]node {$b$} (q_1) 
(q_1) 
edge node {$a$} (q_2) 
(q_2) 
edge  node {$b$} (q_3) 
(q_2) 
edge [bend left] node {$a$} (q_0) 
(q_3) 
edge  node {$b$} (q_4) 
(q_3) 
edge [bend left] node {$a$} (q_2) 
(q_4) 
edge  [bend left]node {$b$} (q_1) 
(q_4) 
edge  node {$a$} (q_5) 
(q_5) 
edge  node {$b$} (q_6) 
(q_5) 
edge [bend left] node {$a$} (q_0) 
(q_6) 
edge  [bend left] node {$b$} (q_4) 
(q_6) 
edge node {$a$} (q_7)
(q_7) 
edge  node {$b$} (q_8) 
(q_7) 
edge [bend left] node {$a$} (q_0)  
(q_8) 
edge  [bend left] node {$b$} (q_4) 
(q_8) 
edge [bend left] node {$a$} (q_2) 









;
\end{tikzpicture} 
\end{center}
Автомат построен по алгоритму, разобранному на занятии.\\
$l(a)=\epsilon$\\
$l(bb)=b$\\
$l(baa)=\epsilon$\\
$l(baba)=ba$\\
$l(babbb)=b$\\
$l(babbaa)=q_0$\\
$l(babbabb)=babb$\\
$l(babbabaa)=\epsilon$\\
$l(babbababa)=ba$\\
$l(babbababb)=babb$\\

2)\\

\begin{center} 
\begin{tikzpicture}[>=stealth',shorten >=1pt,auto,node distance=2cm] 
\node[initial, state] (q_0) {$q_0$}; 
\node[state] (q_1) [below right of=q_0]{$b$}; 
\node[ state] (q_2) [below right of=q_1]{$ba$}; 
\node[ state] (q_3) [below right of=q_2]{$bab$}; 
\node[state] (q_4) [below right of=q_3] {$babb$}; 
\node[state] (q_5) [below right of=q_4] {$babba$}; 
\node[state] (q_6) [below right of=q_5] {$babbab$}; 
\node[state] (q_7) [below right of=q_6] {$babbaba$}; 
\node[state,accepting] (q_8) [below right of=q_7] {$babbabab$}; 

\path[->] (q_0) 
edge  node {$b$} (q_1) 
(q_0) 
edge [loop above] node {$a$} (q_0) 
(q_1) 
edge node {$a$} (q_2) 
(q_2) 
edge  node {$b$} (q_3) 

(q_3) 
edge  node {$b$} (q_4) 

(q_4) 
edge  node {$a$} (q_5) 
(q_5) 
edge  node {$b$} (q_6) 

(q_6) 
edge node {$a$} (q_7)
(q_7) 
edge  node {$b$} (q_8) 
;
\path[->] [draw=green](q_1)  [bend left] edge node {$$} (q_0);
\path[->] [draw=green](q_2)  [bend left] edge node {$$} (q_0);
\path[->] [draw=green](q_3)  [bend left] edge node {$$} (q_1);
\path[->] [draw=green](q_4)  [bend left] edge node {$$} (q_1);
\path[->] [draw=green](q_5)  [bend left] edge node {$$} (q_2);
\path[->] [draw=green](q_6)  [bend left] edge node {$$} (q_3);
\path[->] [draw=green](q_7)  [bend left] edge node {$$} (q_2);
\path[->] [draw=green](q_8)  [bend left] edge node {$$} (q_3);
\end{tikzpicture} 
\end{center}
3)\\
Слово-babbabbabab\\
$(q_0,babbabbabab)$\\
$(b,abbabbabab)$\\
$(ba,bbabbabab)$\\
$(bab,babbabab)$\\
$(babb,abbabab)$\\
$(babba,bbabab)$\\
$(babbab,babab)$\\
$(bab,babab)$\\по ссылке\\
$(babb,abab)$\\
$(babba,bab)$\\
$(babbab,ab)$\\
$(babbaba,b)$\\
$(babbabab,\epsilon)$\\Состояние допускающее, значит слово допустимо\\
Cлово - babbabc\\
$(q_0,babbabc)$\\
$(b,abbabc)$\\
$(ba,bbabc)$\\
$(bab,babc)$\\
$(babb,abc)$\\
$(babba,bc)$\\
$(babbab,c)$\\
по ссылке\\
$(bab,c)$\\
по ссылке\\
$(b,c)$\\
по ссылке\\
$(q_0,c)$\\
Автомат сломался(множество возможных состояний 0), значит слово недопустимо(нет ни одного допустимого состояния)\\
\textbf{2.}\\






\begin{center} 
\begin{tikzpicture}[>=stealth',shorten >=1pt,auto,node distance=2cm] 
\node[initial, state] (q_0) {$q_0$}; 
\node[state] (A) [right of=q_0]{$a$}; 
\node[ state,accepting] (C) [below  of=A]{$c$}; 
\node[ state,accepting] (B) [below of=C]{$b$}; 
\node[state] (AA) [above right of=A] {$aa$}; 
\node[state,accepting] (AC) [below right of=A] {$ac$}; 
\node[state,accepting] (AAC) [ right of=AA] {$aac$}; 
\node[state,accepting] (ACB) [ right of=AC] {$acb$}; 

\path[->] (q_0)  edge node {$a$} (A);
\path[->] (q_0)  edge node {$b$} (B);
\path[->] (q_0)  edge node {$c$} (C);
\path[->] (A)  edge node {$a$} (AA);
\path[->] (A)  edge node {$c$} (AC);
\path[->] (AA)  edge node {$c$} (AAC);
\path[->] (AC)  edge node {$b$} (ACB);


\path[->] [draw=green](A)  [bend left] edge node {$$} (q_0);
\path[->] [draw=green](B)  [bend left] edge node {$$} (q_0);
\path[->] [draw=green](C)  [bend left] edge node {$$} (q_0);
\path[->] [draw=green](AA)  [bend left] edge node {$$} (A);
\path[->] [draw=green](AC)  [bend left] edge node {$$} (C);
\path[->] [draw=green](AAC)  [bend left] edge node {$$} (AC);
\path[->] [draw=green](ACB)  [bend left] edge node {$ $} (B);






\end{tikzpicture} 
\end{center}
Cлово-aacbacb\\
Расчитаем количество допустимых состояний у каждого состояния, до которых можно добраться по суффиксным ссылкам\\
a-0\\
aa-0\\
aac-3\\
ac-2\\
acb-2\\
b-1\\
c-1\\
Теперь опишем работу\\
$(q_0,aacbacb,0)$\\
$([a],acbacb,0)$\\
$([aa],cbacb,0)$\\
$([aac],bacb,3)$\\
$([acb],acb,3+2)$\\
$([a],cb,3+2)$\\
$([ac],b,3+2+2)$\\
$([acb],\epsilon,3+2+2+2)$\\
Итого:9\\
\textbf{3.}\\
Построим ДКА, распознающий подслово aab.\\
Будем строить КМП-автомат и оставаться по достижению конца в допустимом состоянии. Исходя из корректности КМП мы дойдем до допустимого состояния тогда и только тогда, когда мы прочитаем в первый раз подслово aab.\\
$l(q_0,b)=q_0$\\
$l(q_1,b)=q_0$\\
$l(q_2,a)=q_2$\\


\begin{center} 
\begin{tikzpicture}[>=stealth',shorten >=1pt,auto,node distance=2cm] 
\node[initial, state] (q_0) {$q_0$}; 
\node[state] (A) [right of=q_0]{$q_1$}; 
\node[state] (AA) [ right of=A] {$q_2$}; 
\node[state,accepting] (AAB) [ right of=AA] {$q_3$}; 

\path[->] (q_0)  edge node {$a$} (A);
\path[->] (q_0) [loop above] edge node {$b$} (q_0);



\path[->] (A)  edge node {$a$} (AA);
\path[->] (A) [bend left] edge node {$b$} (q_0);
\path[->] (AA)  edge node {$b$} (AAB);
\path[->] (AA)  [loop above]edge node {$a$} (AA);
\path[->] (AAB)  [loop above]edge node {$a,b$} (AAB);







\end{tikzpicture} 
\end{center}


Построим ДКА, распознающий подслово ba\\
Будем строить КМП-автомат и оставаться по достижению конца в допустимом состоянии. Исходя из корректности КМП мы дойдем до допустимого состояния тогда и только тогда, когда мы прочитаем в первый раз подслово ba.\\
$l(q_0,a)=q_0$\\
$l(q_1,b)=q_1$\\

\begin{center} 
\begin{tikzpicture}[>=stealth',shorten >=1pt,auto,node distance=2cm] 
\node[initial, state] (q_0) {$q_0$}; 
\node[state] (B) [right of=q_0]{$q_1$}; 
\node[state,accepting] (BA) [ right of=B] {$q_2$}; 


\path[->] (q_0)  edge node {$b$} (B);
\path[->] (q_0) [loop above] edge node {$a$} (q_0);



\path[->] (B)  edge node {$a$} (BA);
\path[->] (B) [loop above] edge node {$b$} (B);

\path[->] (BA)  [loop above]edge node {$a,b$} (BA);








\end{tikzpicture} 
\end{center}
Хорошо, построим произведение этих автоматов\\
\begin{center} 
\begin{tikzpicture}[>=stealth',shorten >=1pt,auto,node distance=2cm] 
\node[initial, state] (q_00) {$q_{00}$}; 
\node[state] (q_01) [right of=q_00]{$q_{01}$}; 
\node[state] (q_02) [right of=q_01]{$q_{02}$}; 
\node[state] (q_03) [right of=q_02]{$q_{03}$}; 

\node[state] (q_10) [below of =q_00]{$q_{10}$}; 
\node[state] (q_11) [right of=q_10]{$q_{11}$}; 
\node[state] (q_12) [right of=q_11]{$q_{12}$}; 
\node[state] (q_13) [right of=q_12]{$q_{13}$}; 


\node[state] (q_20) [below of =q_10]{$q_{20}$}; 
\node[state] (q_21) [right of=q_20]{$q_{21}$}; 
\node[state] (q_22) [right of=q_21]{$q_{22}$}; 
\node[state,accepting] (q_23) [right of=q_22]{$q_{23}$}; 



\path[->] (q_00)  edge node {$a$} (q_01);
\path[->] (q_00)  edge node {$b$} (q_10);

\path[->] (q_01)  edge node {$a$} (q_02);
\path[->] (q_01)  edge node {$b$} (q_10);

\path[->] (q_02)  [loop above]edge node {$a$} (q_02);
\path[->] (q_02)  edge node {$b$} (q_13);

\path[->] (q_03)  [loop above]edge node {$a$} (q_03);
\path[->] (q_03)  edge node {$b$} (q_13);

\path[->] (q_10)  edge node {$a$} (q_21);
\path[->] (q_10)  [loop left]edge node {$b$} (q_10);

\path[->] (q_11)  edge node {$a$} (q_22);
\path[->] (q_11)  edge node {$b$} (q_10);

\path[->] (q_12)  edge node {$a$} (q_22);
\path[->] (q_12)  edge node {$b$} (q_13);

\path[->] (q_13)  edge node {$a$} (q_23);
\path[->] (q_13)  [loop right]edge node {$b$} (q_13);

\path[->] (q_20)  edge node {$a$} (q_21);
\path[->] (q_20)  [loop left]edge node {$b$} (q_20);

\path[->] (q_21)  edge node {$a$} (q_22);
\path[->] (q_21)  [bend left]edge node {$b$} (q_20);

\path[->] (q_22)  [loop below]edge node {$a$} (q_22);
\path[->] (q_22)  edge node {$b$} (q_23);

\path[->] (q_23)  [loop right]edge node {$a$} (q_23);
\path[->] (q_23)  [loop below]edge node {$b$} (q_23);



\end{tikzpicture} 
\end{center}
Видим, что мы никак не можем попасть в $q_{12},q_{03},q_{11}$\\
Поэтому полученный автомат имеет 9 состояний. \\
Построим ДКА, допускающий подслово babb\\
Будем строить КМП-автомат и оставаться по достижению конца в допустимом состоянии. Исходя из корректности КМП мы дойдем до допустимого состояния тогда и только тогда, когда мы прочитаем в первый раз подслово ba.\\
$l(q_0,a)=q_0$\\
$l(q_1,b)=q_1$\\
$l(q_2,a)=q_0$\\
$l(q_3,a)=q_2$\\



\begin{center} 
\begin{tikzpicture}[>=stealth',shorten >=1pt,auto,node distance=2cm] 
\node[initial, state] (q_0) {$q_0$}; 
\node[state] (B) [right of=q_0]{$q_1$}; 
\node[state] (BA) [ right of=B] {$q_2$}; 
\node[state] (BAB) [ right of=BA] {$q_3$}; 
\node[state,accepting] (BABB) [ right of=BAB] {$q_4$}; 

\path[->] (q_0) [loop above] edge node {$a$} (q_0);
\path[->] (q_0)  edge node {$b$} (B);

\path[->] (B)  [loop above]edge node {$b$} (B);
\path[->] (B)  edge node {$a$} (BA);

\path[->] (BA)  edge node {$b$} (BAB);
\path[->] (BA)  [bend left]edge node {$a$} (q_0);

\path[->] (BAB) edge node {$b$} (BABB);
\path[->] (BAB)  [bend left]edge node {$a$} (BA);

\path[->] (BABB)  [loop above]edge node {$a,b$} (BABB);








\end{tikzpicture} 
\end{center}

Мы умеем делать отрицание полного ДКА, оно будет содержать тоже 5 состояний. Если мы возбьмем произведение этого автомата и автомата сверху, содержащего 9 вершин, то мы как раз получим тот автомат, который нам нужно.(Исходя из корректности конструкции произведения и конструкции отрицания для полных ДКА)\\
Полученный ДКА будет содержать 5*9=45 вершин.\\




\end{document}