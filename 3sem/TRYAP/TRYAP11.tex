
\documentclass[a4paper,12pt]{article} 




\usepackage[utf8]{inputenc}			 
\usepackage[english,russian]{babel}	
\usepackage[all]{xy}

%автомат
\usepackage{tikz}
\usetikzlibrary{arrows,automata}


% Математика
\usepackage{amsmath,amsfonts,amssymb,amsthm,mathtools} 


\usepackage{wasysym}


\usepackage{graphicx}
\graphicspath{{pictures_tryap1/}}



\author{Томинин Ярослав, 778 группа}
\title{Домашнее задание №11}
\date{\today}


\begin{document} 

\maketitle
\newpage
\textbf{1.}\\
Рассмотрим СУ, которая делает перевод $w\to \pi_l(w)$. Она была рассмотрена в примере 1. Построим вывод ((a))в грамматике G. 
\begin{align*}
	E&\to T\\\
	T&\to F\\\
	F&\to (E)\\\
	E&\to T\\\
	T&\to F\\\
	F&\to (E)\\\
	E&\to T\\\
	T&\to F\\\
	F&\to a\\\
\end{align*}
Теперь применим те же правила в грамматике, в которую нас перевел анализатор.\\
\begin{align*}
	E&\to 2T\\\
	T&\to 4F\\\
	F&\to 5E\\\
	E&\to 2T\\\
	T&\to 4F\\\
	F&\to 5E\\\
	E&\to 2T\\\
	T&\to 4F\\\
	F&\to 6\\\
\end{align*}
Получаем левый разбор-результат выполнение правил:245245246\\
Для того, чтобы получить правый разбор, достаточно во всех правилах вывода G' перенести цифру в конец вывода. Тогда получаем \\
\begin{align*}
	E&\to T2\\\
	T&\to F4\\\
	F&\to E5\\\
	E&\to T2\\\
	T&\to F4\\\
	F&\to E5\\\
	E&\to T2\\\
	T&\to F4\\\
	F&\to 6\\\
\end{align*}
Тогда правый разбор: 642542542\\
Или можно сделать все то же самое, построив дерево разбора. В данном случае дерево разбора для левого и правого вывода одинаковое.\\
\begin{center}
\begin{tikzpicture}
\node {$E$} %root
child{ node {$T$}
child{ node{$($} }
child{ node {$E$}
child{ node {$F$}
child{ node {$($}}
child{ node {$E$}
child{ node {$T$}
child{ node {$F$}
child{ node {$a$}}}}}
child{ node {$)$}}}
}
child{ node {$)$}}
};
\end{tikzpicture}

\end{center}
\textbf{2.}\\
\begin{tabular}{ || l | l | l | l | l | l || }
\hline
$F_i$ &E & T & F & E' & T'  \\ \hline
$F_0$ &$\oslash$ & $\oslash$ & $\oslash$ & $\epsilon$ & $\epsilon$  \\ \hline
$F_1$ &$\oslash$ & $\oslash$ & (,id & $\epsilon,+$ & $\times, \epsilon$  \\ \hline
$F_2$ &$\oslash$ & (,id & (,id & $\epsilon,+$ &   $\times, \epsilon$ \\ \hline
$F_3$ &$(,id$ & (,id & (,id & $\epsilon,+$ &   $\times, \epsilon$ \\ \hline
\hline
\end{tabular}
\\
Теперь вычислим FOLLOW\\
Добавим правило $S'\to E\$$\\

\begin{tabular}{  || l | l | l | l | l | l || }
\hline
$F_i$ &E & T & F & E' & T'  \\ \hline
$F_0$ &$\$$ & $\oslash$ & $\oslash$ & $\oslash$ & $\oslash$  \\ \hline
$F_1$ &$\$,)$ & $\$,+$ &$\times$ & $\$$ & $\oslash$  \\ \hline
$F_2$ &$\$,)$ & $\$,+,)$ & $\$,\times,+$ & $\$,)$ &   $\$,+$ \\ \hline
$F_3$ &$\$,)$ & $\$,+,)$ & $\$,\times,+,)$ & $\$,)$ &   $\$,+,)$ \\ \hline
$F_4$ &$\$,)$ & $\$,+,)$& $\$,\times,+,)$ & $\$,)$ &   $\$,+,)$ \\ \hline
$F_5$ &$\$,)$ & $\$,+,)$& $\$,\times,+,)$ & $\$,)$ &   $\$,+,)$ \\ \hline

\hline
\end{tabular}\\
Теперь построим анализатор.\\

\begin{tabular}{  || l | l | l | l | l | l || }
\hline
$0$ &E & T & F & E' & T'  \\ \hline
$id$ &$E\to TE'$&$T\to FT'$&$F\to id$&$NONE$&$NONE$  \\ \hline
$+$ &NONE&$T\to FT'$&NONE&$E'\to +TE'$&$T'\to \epsilon$  \\ \hline
$\times$ &NONE&NONE&NONE&NONE&$T'\to \times FT'$  \\ \hline
$($ &$E\to TE'$&$T\to FT'$&$F\to (E)$&NONE&NONE  \\ \hline
$)$ &NONE&NONE&NONE&$E'\to \epsilon$&$T'\to \epsilon$  \\ \hline
$\$$ &NONE&NONE&NONE&$E'\to \epsilon$&$T'\to \epsilon$  \\ \hline
\hline
\end{tabular}\\
Для  демострации запишем конфигурации\\
Для начала занумеруем правила\\

\begin{align*}
	E\to TE'(1)&\\\
	E'\to +TE'|\epsilon(2,3)&\\\
	T\to FT'(4)&\\\
	T'\to \times FT'|\epsilon(5,6)&\\\
	F\to (E)|id(7,8)&\\\
\end{align*}
\begin{align*}
	&(id+id\times id\$ , E\$,) \\\
	&(id+id\times id\$ , TE'\$,1) \\\
	&(id+id\times id\$ , FT'E'\$,14) \\\
	&(id+id\times id\$ , idT'E'\$,148) \\\
	&(+id\times id\$ ,  T'E'\$,148) \\\
	&(+id\times id\$ , E'\$,1486) \\\
	&(+id\times id\$ , +TE'\$,14862) \\\
	&(id\times id\$ , TE'\$,14862) \\\
	&(id\times id\$ , FT'E'\$,148624) \\\
	&(id\times id\$ , idT'E'\$,1486248) \\\
	&(\times id\$ , T'E'\$,1486248) \\\
	&(\times id\$ , \times FT'E'\$,14862485) \\\
	&(id\$ , FT'E'\$,14862485) \\\
	&(id\$ , idT'E'\$,148624858) \\\
	&(\$ , T'E'\$,148624858) \\\
	&(\$ , E'\$,1486248586) \\\
	&(\$ , \$,14862485863) \\\
	&(\epsilon , \epsilon,14862485863) \\\	
\end{align*}
\\
Ура, это слово принимается анализатором!\\
Дерево разбора:\\
\begin{center}
\begin{tikzpicture}[>=stealth',shorten >=1pt,auto,node distance=3cm] 
\node {$E$} %root
child{ node [above]{$1$}}
child{ node [left]{$T$}
child{ node {$4$}}
child{ node {$F$}
child{ node {$8$}}
}
child{ node {$T'$}
child{ node {$6$}}
}
}
child{node[right]{$E'$}
child{ node {$2$}}
child{ node {$T$}
child{ node {$4$}}
child{ node {$F$}
child{ node {$8$}}
}
child{ node [right]{$T'$}
child{ node {$5$}}
child{ node {$F$}
child{ node {$8$}}
}
child{ node {$T'$}
child{ node {$6$}}
}
}
}
child{ node {$3$}}
};
\end{tikzpicture}

\end{center}
Это дерево сходится с предыдущей последовательностью выводов.\\
\textbf{3.}\\
Построим FIRST\\

\begin{tabular}{  || l | l |l|| }
\hline
$F_i$ & S& A  \\ \hline
$F_0$ & $\oslash$&$\epsilon$  \\ \hline
$F_1$ & $a,b$& $\epsilon,a,b$  \\ \hline
$F_2$ & a,b& $a,b,\epsilon$  \\ \hline
\hline
\end{tabular}\\
Из этой таблицы уже видно, что пересечение FIRST(A) и FOLLOW(A) не пустое множество и содержит b, так как b точно содержится в FOLLOW и мы знаем, что оно содержится в FIRST. Так как $\epsilon \in FIRST(A)$, то не выполняется 2 критерий.\\
Построим $FIRST_2(A), FOLLOW_2(A)$. Для этого воспользуемся алгоритмом\\

\begin{tabular}{  || l | l |l|| }
\hline
$F_i$ & S& A  \\ \hline
$F_0$ & $\oslash$&$b,\epsilon$  \\ \hline
$F_1$ & $aa,ab,bb$& $\epsilon,b$  \\ \hline
$F_2$ &  $aa,ab,bb$& $b,\epsilon$  \\ \hline
\hline
\end{tabular}\\
Теперь построим $FOLLOW_2(A)$. \\
Сделаем грамматику пополненной.\\
$S'\to S\$$\\
\begin{tabular}{  || l | l |l |l || }
\hline
$F_i$ & S& A & S' \\ \hline
$F_0$ & $\$ $&$aa,ba$ &$\oslash$ \\ \hline
$F_1$ & $\$ $&$aa,ba$ & $\oslash$\\ \hline
\hline
\end{tabular}\\ 
Отлично, теперь  проверим выполняется ли критерий. \\
Рассмотрим первые два правила\\
$FIRST_2(aAaa)=FIRST_2(a)\oplus_2 FIRST_2(A)\oplus_2 FIRST_2(a)=\lbrace a\rbrace \oplus_2 \lbrace b,\epsilon\rbrace\oplus_2 \lbrace a\rbrace=\lbrace aa,ab\rbrace$ \\
$FIRST_2(babA)==\lbrace ba\rbrace$ \\
Так как нет пересечений, то здесь нет противоречияя критерию.\\
Рассмотрим последние два правила.\\
$FIRST_2(b\alpha)=FIRST_2(b)\oplus_2 FOLLOW_2(A)=\lbrace a\rbrace \oplus \lbrace aa,ba\rbrace=\lbrace ba,bb\rbrace$\\
$FIRST_2(\epsilon \alpha)=FOLLOW_2(A)= \lbrace aa,ba\rbrace$\\
Заметим, что последние два множества имеют пересечение, поэтому это грамматика не является LL(2)-грамматикой.\\
\textbf{4.}\\
\begin{align*}
	&S\to baaA|babA\\\
	&A\to \epsilon|Aa|Ab\\\
\end{align*}
Преобразуем грамматику\\
\begin{align*}
	&S\to baQ(1)\\\
	&Q\to aA|bA(2,3)\\\
	&A\to A'(4)\\\
	&A'\to aA'|bA'|\epsilon(5,6,7)\\\
\end{align*}
Построим FIRST\\
\begin{tabular}{  || l | l |l |l | l || }
\hline
$F_i$ & S& A&A'& Q  \\ \hline
$F_0$ & $\oslash $&$\epsilon$& $\oslash$&$\oslash$\\ \hline
$F_1$ & $b $&$\epsilon$& $a,b$& $a,b$\\ \hline
$F_2$ & $b $&$\epsilon,a,b$ &$a,b$& $a,b $\\ \hline
\hline
\end{tabular}\\ 
Построим FOLLOW\\
\begin{tabular}{  || l | l |l |l | l || }
\hline
$F_i$ & S& A&A'& Q  \\ \hline
$F_0$ & $\$ $& $\oslash$& $\oslash$&$\oslash$\\ \hline
$F_1$ & $\$ $& $\oslash$& $\oslash$&$\$ $\\ \hline
$F_2$ & $\$ $& $\$ $& $\$ $&$\$ $\\ \hline
$F_3$ & $\$ $&$\$ $ &$\$ $& $\$$\\ \hline
\hline
\end{tabular}\\ 
Построим таблицу анализатора.\\
\begin{tabular}{  || l | l |l |l | l || }
\hline
$0$ & S& A&A'& Q  \\ \hline
$a$ & NONE& 4&5& 2  \\ \hline
$b$ & 1& 4&6& 3  \\ \hline
$\$ $ & NONE& 4&7& NONE  \\ \hline
\hline
\end{tabular}\\ 
Демонстрация с помощью конфигураций.\\
\begin{align*}
	&(baab\$,S\$,)\\\
	&(baab\$,baQ\$,1)\\\
	&(aab\$,aQ\$,1)\\\
	&(ab\$,Q\$,1)\\\
	&(ab\$,aA\$,12)\\\
	&(b\$,A\$,12)\\\
	&(b\$,A'\$,124)\\\
	&(b\$,bA'\$,1246)\\\
	&(\$,A'\$,1246)\\\
	&(\$,\$,12467)\\\
	&(\epsilon,\epsilon,12467)\\\
\end{align*}
Cлово принимается нашей грамматикой.\\
\textbf{5.}\\
\begin{align*}
	& S'\to S\$\\\
	& S\to aSb|a\\\
\end{align*}
Рассмотрим начальное состояние $[S'\to \cdot S\$,\\epsilon] $\\
Проведем поиск в ширину, то есть из достижимых вершин будем проводить все возможные переходы. Закончим, когда не появится ничего нового на k-ой итеррации.\\
\begin{center} 
\begin{tikzpicture}[>=stealth',shorten >=1pt,auto,node distance=4cm] 

\node[initial,state] (1)  {$(1)S'\to \cdot S, \$$}; 
\node[state] (2) [above right  of = 1] {$(2)S'\to \cdot a, \$$}; 
\node[state] (3) [below right  of = 1] {$(3)S'\to \cdot aSb, \$$}; 
\node[state] (4) [right of = 1] {$(4)S'\to S\cdot , \$$}; 

\node[state] (5) [right of = 2] {$(5)S'\to a \cdot, \$$}; 
\node[state] (6) [below of = 3] {$(6)S'\to a \cdot Sb, \$$}; 
\node[state] (7) [above right of = 6] {$(7)S'\to aS \cdot b, \$$}; 
\node[state] (8) [right of = 7] {$(8)S'\to aSb\cdot , \$$}; 
\node[state] (9) [below left of = 6] {$(9)S'\to \cdot aSb, b  $}; 
\node[state] (10) [right of =6] {$(10)S'\to \cdot a, b  $}; 
\node[state] (11) [right of = 9] {$(11)S'\to a\cdot Sb, b  $}; 
\node[state] (12) [right of = 11] {$(12)S'\to aS\cdot b , b  $}; 
\node[state] (13) [right of = 12] {$(13)S'\to aSb\cdot , b  $}; 
\node[state] (14) [right of = 10] {$(14)S'\to a\cdot , b  $}; 

\path[->]  (1) edge node  {$S$}(4);
\path[->]  (1) edge node  {$\epsilon$}(2);
\path[->]  (1) edge node  {$\epsilon$}(3);

\path[->]  (2) edge node  {$a$}(5);

\path[->]  (3) edge node  {$a$}(6);


\path[->]  (6) edge node  {$\epsilon$}(9);
\path[->]  (6) edge node  {$\epsilon$}(10);
\path[->]  (6) edge node  {$S$}(7);

\path[->]  (7) edge node  {$b$}(8);

\path[->]  (9) edge node  {$a$}(11);

\path[->]  (10) edge node  {$a$}(14);

\path[->]  (11) edge [bend right] node  {$\epsilon$}(9);
\path[->]  (11) edge node  {$\epsilon$}(10);
\path[->]  (11) edge node  {$S$}(12);

\path[->]  (12) edge node  {$b$}(13);




\end{tikzpicture} 
\end{center}

Построим ДКА по НКА\\
\begin{tabular}{  || c | c |c |c || }
\hline
$Q_i$ & a& b& S  \\ \hline
$Q_0=\lbrace 1,2,3\rbrace$ &$Q_1=$ 5,6,9,10& -&$Q_2=$4 \\ \hline
$Q_1=\lbrace 5,6,9,10\rbrace$ & $Q_3=$11,10,9,14& -& $Q_4=$7,12  \\ \hline
$Q_2=\lbrace 4\rbrace$ & -& -& -  \\ \hline
$Q_3=\lbrace 9,10,11,14\rbrace$ & $Q_3=$9,10,11,14,& -& $Q_5=$12  \\ \hline
$Q_4=\lbrace 7,12\rbrace$ & -& $Q_6=$8,13& -  \\ \hline
$Q_5=\lbrace 12\rbrace$ & -& $Q_7=$13& -  \\ \hline
$Q_6=\lbrace 8,13\rbrace$ & -& -& -  \\ \hline
$Q_7=\lbrace 13\rbrace$ & -& -& -  \\ \hline


\hline
\end{tabular}\\ 
Эта таблица определяет ДКА.\\
\textbf{6.}\\
Поймем, что в произвольной грамматике, которая порождает наш язык есть два правила вида \\
\begin{align*}
	&S\to \alpha\\\
	&S\to \beta\\\
\end{align*}
Причем мы знаем, что $a^k \in FIRST_k(\alpha)$ и $a^k \in FIRST_k(\beta)$.\\
Если мы докажем этот факт, то мы получим непустое пересечение, следовательно, наша грамматика не будет LL. Так как мы можем сказать это для произвольной грамматика, то наш язык не является LL языком.\\
Поймем, что из S должен выводиться $a^k$, поэтому в одном из FIRST точно содержится  $a^k$. \\
Так же поймем, что у нас точно будет два правила такого вида. Докажем это от противного. Допустим, что существует только одно правило такого вида, тогда наша грамматика бужет допускать слова $a^kb^i$, но этого быть не должно. 
Поэтому у нас есть два правила, где второе правило должно "отвечать " за вывод $a^nb^n$, поэтому в FIRST второго правила тоже содержится $a^k$.\\


\end{document}