
\documentclass[a4paper,12pt]{article} 




\usepackage[utf8]{inputenc}			 
\usepackage[english,russian]{babel}	
\usepackage[all]{xy}

%автомат
\usepackage{tikz}
\usetikzlibrary{arrows,automata}


% Математика
\usepackage{amsmath,amsfonts,amssymb,amsthm,mathtools} 


\usepackage{wasysym}


\usepackage{graphicx}
\graphicspath{{pictures_tryap1/}}



\author{Томинин Ярослав, 778 группа}
\title{Домашнее задание №9}
\date{\today}


\begin{document} 

\maketitle
\newpage
\textbf{1.}\\
\textbf{$G_n$:}\\
\begin{align*}
	S&\to A_1A_2...A_nA_n...A_1\\\
	A_i&\to aA_{i+1} \forall i \in [1,n-1]\\\
	A_n&\to ab\\\
\end{align*}
Докажем, что эта грамматика порождает нужное слово и только его. Заметим, что мы можем убрать нетерминал используя второй тип правил, пока не добъемся $A_n$, а только потом уже убрать $A_n$. Именно поэтому терминал $A_i$ превратится в $a^{n-i+1}b$.(Eще убедительней это можно доказать по индукции по i: для n-очевидно, для i мы можем свести его к i+1 только единственным правилом, поэтому в нем будет на 1 букву а больше, далее используем предположение индукции и получаем то же утверждение для нетерминала с меньшим индексом). Тогда мы можем получить желаемое слово и только его. Ясно, что длина описания cn, на первой строчке сn, на следующих n строчках по константе.\\
\textbf{2.}\\
\textbf{Алгоритм:} поставим в соответствие \\
\textbf{3.}\\
a)\\
\textbf{Построение SLG $G_w$:}\\
\begin{align*}
	S&\to A_1A_2A_3A_4\\\
	A_1&\to a,u=a^7\\\ 
	A_2&\to A_1a,u=a^5\\\
	A_3&\to A_2a,u=a^2\\\
	A_4&\to A_2,u=\epsilon\\\
\end{align*}

\textbf{Построение LZW-автомата:}\\

\begin{center} 
\begin{tikzpicture}[>=stealth',shorten >=1pt,auto,node distance=2cm] 

\node[initial,state] (q_0)  {$q_0$}; 
\node[state] (q_1) [right of = q_0] {$A_1$}; 
\node[state] (q_2) [right of = q_1] {$A_2$}; 
\node[state] (q_3) [right of = q_2] {$A_3$}; 




\path[->]  (q_0) edge [bend left] node 
{$a$}(q_1);
\path[->]  (q_1) edge [bend left] node 
{$a$}(q_2);
\path[->]  (q_2) edge [bend left] node 
{$a$}(q_3);


\end{tikzpicture} 
\end{center}
b)\\
\textbf{Построение SLG $G_w$:}\\
\begin{align*}
	S&\to A_1A_2A_3A_4A_5A_6A_7A_8A_9A_{10}A_{11}A_{12}A_{13}A_{14}A_{15}\\\
	A_1&\to t,u=obeornottobeortobeornot\\\ 
	A_2&\to o,u=beornottobeortobeornot\\\
	A_3&\to b,u=eornottobeortobeornot\\\
	A_4&\to e,u=ornottobeortobeornot\\\
	A_5&\to A_2r,u=nottobeortobeornot\\\
	A_6&\to n,u=ottobeortobeornot\\\
	A_7&\to A_2t,u=tobeortobeornot\\\
	A_8&\to A_1o,u=beortobeornot\\\
	A_9&\to A_3e,u=ortobeornot\\\
	A_{10}&\to A_5t,u=obeornot\\\
	A_{11}&\to A_2b,u=eornot\\\
	A_{12}&\to A_4o,u=rnot\\\
	A_{13}&\to r,u=not\\\
	A_{14}&\to A_6o,u=t\\\
	A_{15}&\to A_1,u=\epsilon\\\
\end{align*}

\textbf{Построение LZW-автомата:}\\

\begin{center} 
\begin{tikzpicture}[>=stealth',shorten >=1pt,auto,node distance=2cm] 

\node[initial,state] (q_0)  {$q_0$}; 
\node[state] (q_1) [right of = q_0] {$A_1$}; 
\node[state] (q_2) [above of = q_1] {$A_2$}; 
\node[state] (q_3) [above of = q_2] {$A_3$}; 
\node[state] (q_4) [below of = q_1] {$A_4$}; 
\node[state] (q_5) [above right of = q_2] {$A_5$}; 
\node[state] (q_6) [below of = q_4] {$A_6$}; 
\node[state] (q_7) [below right of = q_2] {$A_7$}; 
\node[state] (q_8) [right of = q_1] {$A_8$}; 
\node[state] (q_9) [right of = q_3] {$A_9$}; 
\node[state] (q_10) [right of = q_5] {$A_{10}$}; 
\node[state] (q_11) [right of = q_2] {$A_{11}$}; 
\node[state] (q_12) [right of = q_4] {$A_{12}$}; 
\node[state] (q_13) [below of = q_6] {$A_{13}$}; 
\node[state] (q_14) [right of = q_6] {$A_{14}$}; 





\path[->]  (q_0) edge  node 
{$t$}(q_1);
\path[->]  (q_0) edge node 
{$o$}(q_2);
\path[->]  (q_0) edge  node 
{$b$}(q_3);
\path[->]  (q_0) edge  node 
{$e$}(q_4);
\path[->]  (q_2) edge  node 
{$r$}(q_5);
\path[->]  (q_0) edge  node 
{$n$}(q_6);
\path[->]  (q_2) edge  node 
{$t$}(q_7);
\path[->]  (q_1) edge  node 
{$o$}(q_8);
\path[->]  (q_3) edge  node 
{$e$}(q_9);
\path[->]  (q_5) edge  node 
{$t$}(q_10);
\path[->]  (q_2) edge  node 
{$b$}(q_11);
\path[->]  (q_4) edge  node 
{$o$}(q_12);
\path[->]  (q_0) edge  node 
{$r$}(q_13);
\path[->]  (q_6) edge  node 
{$o$}(q_14);









\end{tikzpicture} 
\end{center}

\textbf{4.}\\
\begin{align*}
	S&\to A_2A_1A_2,3\\\
	A_1&\to tobeor,6\\\ 
	A_2&\to A_1not,4\\\
\end{align*}
Всего 13 символов. Изначально было 38. Аналогично 1 заданию можно доказать что, из $A_1$ можно породить tobeor и только его(по построению), а из $A_2$ можно породить tobeornot и только его(следует из предыдущего утвержения и построения грамматики).\\ 


\textbf{5.}\\
\textbf{1)}\\
Пункт а покрывается пунктом b, поэтому я буду решать 2 пункта.\\
b)\\
Если решим систему, то мы как раз найдем минимальное по включению решение\\
$X=((110)^*+111^*)^*$
в)\\
\textbf{Общее решение}:$X=(L')^*(\epsilon+L)$,где L  произвольный язык, а L'-язык, содержащий $((110)^*+111^*)$\\
\textbf{Внимание, для слова существуют такие L,L', то оно является решением, я не утверждаю, что если существует L,L' для слова, которые не подходят под мои условия, то это слово не является решением(так как у него может быть подходящая пара L,L')}
Докажем, что любое решение можно выразить так. От противного: допустим, что это не так, тогда найдется решение, в котором L' не содрежит $((110)^*+111^*)$. Но после замыкания X (рекурсивного повторения операции вывода) X будет иметь вид $X=((110)^*+111^*)^*(L')(\epsilon+L)$, что противоречит предположению, так как теперьL' содержит $((110)^*+111^*)^*$.\\
Докажем, что все, что все множнство содержит только решения, достаточно доказать, что $\alpha^iL'(\epsilon+L)=L'(\epsilon+L)$ . Но, должно быть понятно, что 1 содержится во 2 в силу ограничения L'. А 2 содержится в 1 так как мы можем всегда из $\alpha$  выбрать $\epsilon$.\\
\textbf{2)}\\
b)\\
\begin{align*}
	X=(00+01+10+11)^*(0+1+\epsilon)\\\
\end{align*}
в)\\
Так как это весь алфавит, то это и есть общее решение.\\
\textbf{3)}\\
b)\\
\begin{align*}
	Q_0&=0^*(1Q_1+\epsilon)\\\
	Q_1&=10^*1Q_1+10*+0Q_2\\\
	Q_1&=(10^*1)^*0Q_2\\\
	Q_2&=(0(10^*1)^*0+1)Q_2\\\
	Q_2&=(0(10^*1)^*0+1)^*\\\
	Q_1&=(10^*1)^*0(0(10^*1)*0+1)^*\\\
	Q_0&=0^*(1(10^*1)^*0(0(10^*1)^*0+1)^*+\epsilon)\\\
\end{align*}
в)\\
\textbf{6.}\\
\begin{center} 
\begin{tikzpicture}[>=stealth',shorten >=1pt,auto,node distance=4cm] 

\node[initial,state] (q_0)  {$q_0$}; 
\node[state,accepting] (q_1) [right of = q_0] {$q_1$}; 
\node[state,accepting] (q_2) [above of = q_1] {$q_2$}; 
\node[state] (q_3) [right of = q_2] {$q_3$}; 



\path[->]  (q_0) edge [loop below] node 
{$a,a//aa;   a,b//ab;   b,a//ba;   b,b//bb;   a,z//aQz;   b,z//bQz;   $}(q_0);

\path[->]  (q_0) edge  [bend left]node {$c,a//a;   c,b//b;         c,z//Qz;$} (q_1);

\path[->]  (q_1) edge [loop right] node 
{$a,a//\epsilon;       b,b//\epsilon;    $}(q_1);

\path[->]  (q_1) edge [bend right] node 
{$b,a//\epsilon;       a,b//\epsilon;            $}(q_2);

\path[->]  (q_2) edge [loop above] node 
{$b,a//\epsilon;       a,b//\epsilon;    a,a//\epsilon;    b,b//\epsilon;    a,z//z;    b,z//z;$}(q_2);

\path[->]  (q_1) edge [bend right] node 
{$\epsilon,Q//\epsilon;           $}(q_3);


\path[->]  (q_3) edge [bend right] node 
{$a,z//\epsilon;           b,z//\epsilon;$}(q_2);











\end{tikzpicture} 
\end{center}
\textbf{Утверждение}:данный автомат удовлетворяет условию задачи.\\
\textbf{Будем говорить, что слово - паллиндром, если оно имеет следующий вид:$w=xcx^R$}
\textbf{План доказательства}: \\
\textbf{1. После перехода в 1 состояние в нашем стеке будет лежать символы zQ и слово x(если смотреть снизу вверх)}\\
\textbf{2. Находясь в 1 состоянии нет нарушений "паллиндромности" слова}\\
\textbf{3.При переходе в 3 состояние прочитанное слово имеет вид $w=xcx^R$ и если слово имеет вид $w=xcx^R$, то оно в 3 состоянии}\\
\textbf{4. Находясь в состоянии 2, мы уверены, что слово не является паллиндромом и если есть нарушение паллиндромности, то слово находися в 2}\\
\textbf{Доказательство 1:}При считывании букв из х мы будем добавлять в стек каждую букву, так как мы можем перейти в 1 состояние только по с, то на данный момент в стеке будут лежать буквы, соответствующие нашему утверждению(если слово х не пустое). Если же слово х пустое, то при переходе по с мы добавим в стек нужный символ и этот случай тоже будет подходять под наше утверждение.\\
\textbf{Доказательство 2:}Так как мы можем попасть в 1 состояние только из 0, то выполняется утверждение 1. Так как для того, чтобы остаться в этом состоянии, верхний символ стека должен соответствовать прочитанному нами символу, то это означает, что если мы еще не перешли из этого состояния в никакое другое, то не было нарушения паллиндромности.\\
\textbf{Доказательство 3:} В 3 состояние мы можем попасть только из 1, исходя из 2 утверждения и того факта, что мы прочитали Q, мы можем утверждать, что не было нарушения паллиндромности и мы прочитали все слова, следовательно, выполняется утверждение 3.\\
\textbf{Доказательство 4:}Попасть во 2 состояние мы можем либо из первого: тогда у нас есть ошибка паллиндромности(так как, когда мы находились в 1 ее не было в силу 2 утверждения, а перейти во 2 мы могли лишь тогда, когда прочитанный символ не равен верхнему символу в стеке); или из 3 состояния, но тогда $|y|>|x|$(в силу 3 утверждения).\\
Так как, приходя во 2 состояние, мы будем переходить в него же до тех пор, пока слово не закончится(так как определены переходы в случаях, когда в стеке лежит а,b,z, а z всегда лежит в стеке). Поэтому, в силу того, что в 1 наше слово не содержит нарушений, но не является паллиндром и, что во второе состояние мы попадаем тогда и только тогда, когда у нас есть нарушение, этот автомат распознает заданный язык.\\
\textbf{7.}\\
\textbf{8.}\\
Сначала удалим бесплодные символы.\\

\begin{align*}
	V_0&=a,b\\\
	V_1&=a,b,A,B,C\\\
	V_2&=a,b,A,B,C,S,F,E\\\
\end{align*}
Следовательно, G- бесплодный символ.\\
После его удаления, грамматика будет иметь вид:
\begin{align*}
	S&\to A|B|C|E\\\
	A&\to C|aABC|\epsilon\\\
	B&\to bABa|\epsilon\\\
	C&\to BaAbC|\epsilon\\\
	F&\to aBaaCbA\\\
	E&\to A\\\
\end{align*}
Удалим недостижимые.\\
\begin{align*}
	V_0&=S\\\
	V_1&=S,A,B,C,E\\\
	V_2&=S,A,B,C,E\\\
\end{align*}
F- недостижимо. Удалим его и получим приведенную грамматику:
\begin{align*}
	S&\to A|B|C|E\\\
	A&\to C|aABC|\epsilon\\\
	B&\to bABa|\epsilon\\\
	C&\to BaAbC|\epsilon\\\
	E&\to A\\\
\end{align*}
\textbf{9.}\\
Построим КС-грамматику по алгоритму:\\
1)Добавим правила вывода:\\
\begin{align*}
	S&\to [qZ_0q]|[qZ_0p]\\\
\end{align*}
2)рассмотрим переход $(q,X)\in \delta(q,a,Z_0)$ и добавим соответствующие правила вывода
\begin{align*}
	[qZ_0q]&\to a[qXq]\\\
	[qZ_0p]&\to a[qXp]\\\
\end{align*}
3)рассмотрим переход $(q,XX)\in \delta(q,a,X)$ и добавим соответствующие правила вывода
\begin{align*}
	[qXq]&\to a[qXq][qXq]\\\
	[qXq]&\to a[qXp][pXq]\\\
	[qXp]&\to a[qXq][qXp]\\\
	[qXp]&\to a[qXp][pXp]\\\
\end{align*}
4)рассмотрим переход $(q,XX)\in \delta(q,b,X)$ и добавим соответствующие правила вывода
\begin{align*}
	[qXq]&\to b[qXq][qXq]\\\
	[qXq]&\to b[qXp][pXq]\\\
	[qXp]&\to b[qXq][qXp]\\\
	[qXp]&\to b[qXp][pXp]\\\
\end{align*}
5)рассмотрим переход $(p,X)\in \delta(q,b,X)$ и добавим соответствующие правила вывода
\begin{align*}
	[qXq]&\to b[pXq]\\\
	[qXp]&\to b[pXp]\\\
\end{align*}
6)рассмотрим переход $(p,\epsilon)\in \delta(p,a,X)$ и добавим соответствующие правила вывода
\begin{align*}
	[pXp]&\to a\\\
\end{align*}
7)рассмотрим переход $(p,\epsilon)\in \delta(p,b,X)$ и добавим соответствующие правила вывода
\begin{align*}
	[pXp]&\to b\\\
\end{align*}
В результате получим:\\
\begin{align*}
	S&\to [qZ_0q]|[qZ_0p]\\\		
	[qZ_0q]&\to a[qXq]\\\
	[qZ_0p]&\to a[qXp]\\\		
	[qXq]&\to a[qXq][qXq]\\\
	[qXq]&\to a[qXp][pXq]\\\
	[qXp]&\to a[qXq][qXp]\\\
	[qXp]&\to a[qXp][pXp]\\\		
	[qXq]&\to b[qXq][qXq]\\\
	[qXq]&\to b[qXp][pXq]\\\
	[qXp]&\to b[qXq][qXp]\\\
	[qXp]&\to b[qXp][pXp]\\\
	[qXq]&\to b[pXq]\\\
	[qXp]&\to b[pXp]\\\
	[pXp]&\to a\\\
	[pXp]&\to b\\\
\end{align*}
Удалим бесплодные символы
\begin{align*}
	V_0&=a,b\\\
	V_1&=a,b,[pXp]\\\
	V_2&=a,b,[pXp],[qXp]\\\
	V_3&=a,b,[pXp],[qXp],[qZ_0p]\\\
	V_4&=a,b,[pXp],[qXp],[qZ_0p],S\\\
\end{align*}
После удаления бесплодных
\begin{align*}
	S&\to [qZ_0p]\\\		
	[qZ_0p]&\to a[qXp]\\\		
	[qXp]&\to b[pXp]\\\
	[qXp]&\to b[qXp][pXp]\\\
	[qXp]&\to a[qXp][pXp]\\\
	[pXp]&\to a\\\
	[pXp]&\to b\\\
\end{align*}
Найдем недостижимые
\begin{align*}
	V_0&=S\\\
	V_1&=S,[qZ_0p]\\\
	V_2&=S,[qZ_0p],[qXp]\\\
	V_3&=S,[qZ_0p],[qXp],[pXp]\\\
\end{align*}
Все достижимы, поэтому получаем:
\begin{align*}
	S&\to [qZ_0p]\\\		
	[qZ_0p]&\to a[qXp]\\\		
	[qXp]&\to b[pXp]\\\
	[qXp]&\to b[qXp][pXp]\\\
	[qXp]&\to a[qXp][pXp]\\\
	[pXp]&\to a\\\
	[pXp]&\to b\\\
\end{align*}


\end{document}